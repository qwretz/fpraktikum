\documentclass[]{article}
\usepackage{graphicx}
\usepackage{hyperref}
\usepackage{amsmath}
\usepackage{caption}
\usepackage{subcaption}
\usepackage{float}

%opening
\title{Energy Gap in GaAs and Si}
\author{Gunther T\"urk, Jonas Lehnen}

\begin{document}

\maketitle
\begin{abstract}
In this experiment we want to measure the energy gaps for the semiconductors silicon Si and gallium arsenide GaAs. Thereby we want to see the difference between direct and indirect band transfer. To get appropriate values, calibration of photo diode and wavelength values have to be done in advance. In between the lamp spectrum has to be monitored to react to changes in our data caused by the temperature. In the end we want to calculate values for the energy gaps and want to see how the absorption of photons differentiates between the type of transfer.

\end{abstract}

\tableofcontents

\newpage
\section{Theory}
\subsection{Crystallography}
There are a few different types of crystals, depending on their inner structure, they're classified into the following groups. 

The usual crystal, also called mono-crystal, is what one expects when talking about this subject. The atoms in the structure are arranged evenly as shown in >>ref<<. Every atom has its specific place and the distance between the is always the same. The classical example would be the structure of carbon in a diamond, but also semi-conductors without doping are arranged like this, due to their same amount of valence electrons. For Si and GaAs this distance is around $55\ nm$.

The polycrystalline structure is nearly a mono-crystal, but there are some spots where the connection between the atoms is broken. They are many crystals without regularity of shape and orientation. The same process can happen for a high concentration of colloids in a liquid. They can form poly-crystal like structures inside the medium they are floating in. Depending on temperature and mechanical forces, like shaking the sample, it is possible to see the crystals move in the liquid.

\begin{figure}[H]
\centering
\begin{subfigure}[h]{0.59\textwidth}
\includegraphics[width=1\textwidth]{Plots/crystalline.png}
\end{subfigure}
\begin{subfigure}[h]{0.39\textwidth}
\includegraphics[width=1\textwidth]{Plots/lattice.png}
\end{subfigure}
\caption{Visualization of different solid states in the left. Lattice structure of a perfect crystal on the right. \cite{wiki}}
\end{figure}

%%%%% vll ist colloids ein schlechts bsp, da es kein sollid ist

The last class are the amorphous solid bodies. In this state there is no structure in how the atoms are arranged. Most of the man-made solid bodies are considered amorphous. For example glass, polymers and thin films created by sputtering. 

\subsection{Band structure}
The band structure is derived from the Fermi-Dirac distribution. For fermions, spin=1/2 particles, there is a finite amount of how many are allowed to be in the same energy level. This results from Pauli's exclusion principle, where it's not possible to have 2 fermions with the exact same wave function. At a temperature $T=0K$ the electrons energy has to be the lowest possible. And thereby all states underneath the Fermi energy $E_F$ are occupied.

\begin{figure}[H]
\centering
\begin{subfigure}[b]{1\textwidth}
	\centering
	\begin{equation}
	Fermi-Dirac \ statistics: \:\: W(E) = \frac{1}{exp\left(\frac{E-E_F}{k_B T}\right)+1} 
	\end{equation}
\end{subfigure}
\begin{subfigure}[b]{1\textwidth}
	\centering
	\includegraphics[width=.7\textwidth]{Plots/fd.png}
\end{subfigure}
\caption{Fermi-Dirac statistics equation on top and plots for different temperatures underneath, while $T_1=0K$. \cite{wiki}}
\end{figure}

Considering many Coulomb potentials in a one-dimensional lattice, their energy levels overlap and shift each other apart, if they are in the same state, due to Pauli and the electromagnetic interaction between the electrons. These shifted discrete levels are still close to each other and can be merged into a band with a thickness regarding to the energy. The first band in which the electrons are no longer bound to a nucleus is called the conduction band. If an electron is this high-energetic it can be used for a current through the solid. The last band in which the electrons are still bound is called valence band. "Still bound" means that their energy is to low to leave the potential, but they are able to change the nuclei by tunneling. This is not defined as current, because the probability for tunneling into another is as low as the reversed one. The average effect is zero.

There are now three classes of solids, regarding their energy distance between valence and conduction band. This is called the energy gap. As one expects, in a metal there is no gap, electrons are free to be used a current and can change their nucleus like they want. Insulators are the extreme opposite. Their energy gap is so high, around $5eV$, that high voltage is necessary to excite them into the conduction band. This is would usually break the insulator, due to lost power as heat. 
For semi-conductors the energy gap is just less than $1eV$. This means it is possible to excite them by applying a voltage and thereby creating conductivity.

\begin{figure}[H]
\centering
\includegraphics[width=.7\textwidth]{Plots/bandstructure.png}
\caption{Schematic illustration of different types of metal. White: conduction band. Black: valence band. Difference between the bands is equal to the energy gap. The Fermi energy is exactly between the possible occupation levels. \cite{wiki}}
\label{fig:bandstructure}
\end{figure}

Besides a dependency on temperature, the energy gap can change with the process of doping. This considers adding different atoms into a semi-conducting material. Most of the time atoms with three or five valence electrons are used. This results in p- and n-doped versions. In the band structure of a n-doped semi-conductor an additional energy level is created underneath the conduction band, due to an additional valence electron which has no partner to form a covalent bond. This electron is only bond to its nucleus and therefore easily excitable with less energy.
In a p-doped version an additional energy level above the valence band is created. The missing electron for a covalent bond can be replaced by one of the lattice forming ones. Form the foreign atom the electron can be easier excited, due to less binding energy. This means a electron can take two steps, with each lesser energy than the energy gap, to get excited into the conduction band.

%%% pn junction ?? Wird im Versuch nicht benötigt... überfüllt nur, odwer?

\subsection{Band transfer}
For temperatures $T\neq 0$ atoms are oscillating due to their thermal energy. It is possible to quantize this energy analogue to a photons energy. Those lattice oscillations are called phonons. Just like the photons they are described by momentum $\hbar K$ and energy $\hbar \Omega_K$. This energy has to be considered while calculating the energy gap, due to different types of band transfer.

\begin{figure}[H]
\centering
\includegraphics[width=.8\textwidth]{Plots/transfer.png}
\caption{Illustration of direct and indirect band transfer. \cite{wiki}}
\label{fig:band transfer}
\end{figure}

The obvious direct band transfer is not depending on any phonon influence. Maximum of valence and minimum of conduction band are at the same wave vector. This means the excitation happens without a change of momentum and therefore only the energy gap has to be overcome. 

In materials with an indirect band gap, like silicon, the valence maxima and conduction minima are not at the same momentum of the atoms electron. In this case the energy gap is still added by the photons energy, but a change in the lattice momentum is necessary. This happens by emitting or absorbing a phonon, because photons can't change the oscillation of the crystal in such a scale. This means the probability of excitation into the conduction band is less than in the direct transfer. Reason behind this is the necessary temperature, without no phonons could exist. Another angle of view on the function of phonon is to compare them with a catalyser, who excites the electron and afterwards keeps on oscillating. 

%%%%% letzter Satz: passt das als verglecih?

Light Emitting Diodes are usually produced from materials with direct band gap. On this way the intensity of light emitted is higher and more consistent.
%%% indirect

\subsection{Lambert-Beer law}
The Lambert-Beer law describes the relation between absorbed light passing through a solid material, depending on its thickness $dx$ and the materials absorption coefficient $\alpha$. This coefficient also describes the   product of photon absorbing centres per volume $N$ and their effective cross section $\sigma$ to absorb. 

\begin{equation}
-dI(x) = \alpha \cdot I(x)\cdot dx\ \rightarrow\ \alpha=\frac{ln(I_0 / I(x))}{x} = N\cdot \sigma
\end{equation}


\section{Experiment}
\subsection{Set-up}
Via a monochromator only light of a certain wavelength can be filtered. This wavelength will later determine the energy of the light. By truning a screw on the monochromator the reflection lattice is turned and the scale should display the wavelength which is returned. If the lights energy is sufficient the semiconductor will be able to absorb it and in this case there will be no intensity noticeable at our photo diode. 

\begin{figure}[H]
\centering
\includegraphics[width=.9\textwidth]{Plots/setup.png}
\caption{Set-up for this experiment. (1) Lamp, (2) Monochromator, (3) Photo diode, C Cryostat, P1/P2 Polarisation filter, F Coloured filter. }
\label{fig:setup}
\end{figure}

There are four different set-ups used in this experiment. All of them are consisting of a lamp (1), a monochromator (2) and a photo diode (3) arranged as always shown in figure \ref{fig:setup}. In between the part, lenses are placed to keep a focused light path.

In the first task the polarisation filters P1/P2 were placed as shown. Thereby we want to show that with linear increasing the light intensity, the photo-current increases linear, too. This ensures that the photo diode registers all of the incoming light.

Task 2) consists of measuring the intensity while coloured filters F are in the lights path. With this we will compare the values given on the monochromator to the Gaussian peaks in the intensity. This leads to a correction function and more precise wavelengths to calculate the energy gaps.

The third set-up includes the cryostat C with included samples of Si and GaAs. Taking a full lamp spectrum from $400nm \ - \ 1200nm$ we can then see where the photo diode starts measuring light again. This energy will be equivalent to the gap. Once without and once with liquid nitrogen filled in the tank on top to cool the samples to $T=77K$.

Before the normal, in between normal and cold measurement and at the end additional lamp spectra are taken. This happens without any of the third components in figure \ref{fig:setup}. Thereby we want to observe if the conditions of the different energy gap measurements are constant and where to expect to see the photo-current. 

\subsection{Photo diode}
\subsection{Coloured filters} \label{color filters}
Now we want to see if the values the screw on the monochromator displays us is the same wavelength the coloured filters are supposed to let through. Supposing the filters information about the wavelength is more reliable than a screw with mechanical settings.

Due to the given value on the filters we checked 30 surrounding wavelengths for each filter to find each maximum. The Gaussian fits were made with the function:
\begin{equation}
f(x) = A \cdot e^{-\frac{(x-\mu)^2}{2 \sigma^2}} + b
\end{equation}

\begin{figure}[H]
\centering
\includegraphics[width=.9\textwidth]{Plots/905nm-Filter.png}
\caption{Fit for the wavelength calibration at 905nm. Green line indicates the value given on the filters. The other filters are shown in Chapter \ref{Appendix} Appendix.}
\end{figure}

\begin{table}[H]
	\centering
	\begin{tabular}{c|c|c|c}
	Filter [nm] & 768 & 905 & 1060 \\ \hline
	Fit [nm] & 764.92(6) & 899.55(5) & 1071.10(13) \\ \hline
	Difference [nm] & -3.08 & -5.45 & 11.10
	\end{tabular}
	\caption{Comparison of expected and measured maximum intensity wavelength. The filter value is the more trustful to be correct.}
\end{table}

We can now determine how we have to change the wavelengths for the following measurements, by interpolating the values with a straight line:

\begin{figure}[H]
\centering
\includegraphics[width=.9\textwidth]{Plots/LambdaCorrection.png}
\caption{Correction of the monochromator values, due to the measurements with the coloured filters.}
\label{fig:LambdaCorrection}
\end{figure}

For the calculation of the energy gaps, we now have to consider changing the measured values to correct the difference between filters and monochromator. This follows the linear equation used for the fit: 
\begin{equation}
\lambda_{real} = a\cdot \lambda_{measured} \ + \ b
\end{equation}

\subsection{Lamp spectrum} \label{lamp spectrum}
In the case of changes relating to light of our lamp during the time of the experiment, we took the whole lamp spectrum before the first, in between and after the second measurement with the semiconductors in the light beam. It is possible, due to changes in temperature, that the intensities could fluctuate. This would affect the determination of the energy gap. If the plateau area is decreased, the edge of the spectrum could be broader. For the sample on the right of the cryostat, see <<<ref>>>>, this could increase the difficulty to determine a specific wavelength.

But in our case nothing like this happened, as shown in figure \ref{fig:lamp spectra}, where the wavelengths are not yet calibrated with the $2\%$ from section \ref{color filters}.

\begin{figure}[H]
\centering
\includegraphics[width=.8\textwidth]{Plots/All-Lamp-Spectra.png}
\caption{All three measurements of the whole lamp spectrum. The measurements in the legend is related to the following measurements of the energy gaps at different temperatures. }
\label{fig:lamp spectra}
\end{figure} 

As you can see the intensities are dropping rapidly after  $\lambda =1000nm$, for each spectrum. Except of some divergences, which could be caused by rapidly changing the wavelength, they are nearly the identical. The exact measured values are shown in Chapter \ref{Appendix} Appendix.

This concludes in constant conditions for the energy gap measurements for silicon an gallium arsenide in the next chapter.

\subsection{Energy Gaps}

%%% Thickness of sample -> constant and absorbed in A in the  fit eqzuation
\subsection{Conclusion} \label{Conclusion}
In the end we want to talk about error sources. As shown in chapter \ref{lamp spectrum} the spectrum seemed to be constant as well as the room temperature. The position if the cryostat wasn't the same for different temperatures. This could affect the thickness of the sample, but we tried to position it vertical to the lights path. 

In the script it was mentioned to take care of nitrogen clouds and condensation at the cryostats window. These effects could cause decreased intensities measured by the photo diode. By covering the nitrogen with aluminium foil and heating the windows, their influence should be minimized and not noticeable.

In section \ref{color filters} it was strange that the $1060nm$ filter show a displacement to the left instead of the right like the other. The reason for this result seems to be the rapidly dropping intensity after the $\lambda =1000nm$, as discussed in section \ref{lamp spectrum}. After all only 3 filters to calibrate the wavelength seems to be insufficient for excellent results.

The worst error source is the monochromator screw. Although the scale should be correct, the measurements with the filters corrected it. The display of the scale itself seemed to be uncertain on $0.5nm$, because it was possible to turn the screw for a bit without changing the number displayed on the scale. 


\newpage
\section{Appendix} \label{Appendix}
\begin{figure}[H]
\centering
\begin{subfigure}[b]{0.9\textwidth}
\includegraphics[width=\textwidth]{Plots/768nm-Filter.png}
\end{subfigure}
\begin{subfigure}[b]{0.9\textwidth}
\includegraphics[width=\textwidth]{Plots/1060nm-Filter.png}
\end{subfigure}
\caption{Other filter calibrations from section \ref{color filters}. }
\end{figure}


% Tabellen der ganzen dummen Werte


\newpage
\begin{thebibliography}{}

\bibitem{wiki} \begin{verbatim}
http://wiki-fp.physik.uni-mainz.de/index.php/44_energy_gap_in_GaAs_and_Si
\end{verbatim} 
(Begleitendes Skript zum Versuch 44)

\end{thebibliography}
\end{document}


\documentclass[]{article}
\usepackage{graphicx}
\usepackage{hyperref}
\usepackage{amsmath}
\usepackage{caption}
\usepackage{subcaption}
\usepackage{float}

%opening
\title{Energy Gap in GaAs and Si}
\author{Gunther T\"urk, Jonas Lehnen}

\begin{document}

\maketitle
\begin{abstract}
In this experiment we want to measure the energy gaps for the semiconductors silicon Si and gallium arsenide GaAs. Thereby we want to see the difference between direct and indirect band transfer

\end{abstract}

\tableofcontents


\section{Theory}
\subsection{Crystallography}
There are a few different types of crystals, depending on their inner structure, they're classified into the following groups. 

The usual crystal, also called mono-crystal, is what one expects when talking about this subject. The atoms in the structure are arranged evenly as shown in >>ref<<. Every atom has its specific place and the distance between the is always the same. The classical example would be the structure of carbon in a diamond, but also semi-conductors without doping are arranged like this, due to their same amount of valence electrons. For Si and GaAs this distance is around $55\ nm$.

The polycrystalline structure is nearly a mono-crystal, but there are some spots where the connection between the atoms is broken. They are many crystals without regularity of shape and orientation. The same process can happen for a high concentration of colloids in a liquid. They can form poly-crystal like structures inside the medium they are floating in. Depending on temperature and mechanical forces, like shaking the sample, it is possible to see the crystals move in the liquid.

%%%%% 1. Grafiken ausm wiki, 2. vll ist colloids ein schlechts bsp, da es kein sollid ist

The last class are the amorphous solid bodies. In this state there is no structure in how the atoms are arranged. Most of the man-made solid bodies are considered amorphous. For example glass, polymers and thin films created by sputtering. 

\subsection{Band structure}
The band structure is derived from the Fermi-Dirac distribution. For a fermion, spin=1/2, there are


\subsection{Lambert Beer Law}


\section{Experiment}
\subsection{Set-up}

Via a monochromator only light of a certain wavelength can be filtered. This wavelength will later determine the energy of the light. If the lights energy is sufficient the semiconductor will be able to absorb it. In this case there will be no intensity noticeable at our photo diode. 


\section{Appendix}


\newpage
\begin{thebibliography}{}


\end{thebibliography}
\end{document}


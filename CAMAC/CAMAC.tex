\documentclass[]{article}
\usepackage{graphicx}
\usepackage{hyperref}
\usepackage{amsmath}
\usepackage{caption}
\usepackage{subcaption}
\usepackage{float}


%opening
\title{VME Data Acquisition and Plastic Scintillators}
\author{Gunther T\"urk, Jonas Lehnen}

\begin{document}

\maketitle
\begin{abstract}


\end{abstract}

\tableofcontents

\newpage
\section{Theorie}








Problem 4 

Photons with 3 MeV energy are counted by a NaI(TI) detector. Sketch the expected spectrum and explain what you sketch and the physics process beneath it. 


Problem 6 

How many counts are needed to make the standard deviation equal to 1%? And if one obtains 8423456 counts in 300 seconds, what is the standard deviation of the count rate? 
10000
0,6\%

\subsection{Radiation}\label{radiation}

\subsubsection{Isomeric transition}
\subsubsection{Internal Conversion}
\subsubsection{Gamma decay}
%\subsubsection{$\beta^-$ decay}
%\subsubsection{interaciton of $\gamma$-rays in matter}

%% Gunther: der Math mode in den Überschriften jammert manchmal rum... manchmal macht ers aber...

%%% describe 207 Bismuth - decay modes

\subsection{Scintillator}\label{scintillator}
A scintillator is a material that exhibits light, when excited by ionizing radiation. In our experiment we use a two meter plastic scintillator with a photomultiplier at each end to detect the decay of a Bismuth source and background radiation.

When ionizing radiation hits a scintillator it excites high energy molecule states, which then decay in multiple steps back into the ground state. Thereby they emit UV-light. This UV-light is then absorbed again by a wavelength shifter that works like the scintillator but for UV-light and that emits light in the visible spectrum. This can then be measured by the photomultiplier. Because one high energy particle can excite many high energy states until its energy is to low, the amplitude of UV light for each event is proportional to the energy of the ionizing particle.
%Haben wir photomultiplier schonmal irgendwo erklärt? das müsste sonst hier noch dran
\subsection{Photomultiplier}
If a low energy photon hits the photo cathode it ejects an electron through the photo effect. This electron is then accelerated towards several Dynodes in succession with an electric field. Every time an electron hits a dynode, it ejects several electrons from it, due to its higher energy. This leads to an exponential increase in electrons, that can then be measured as a voltage over a resistor. For this process to work the dynodes need to be on an increasingly positive potential. This is usually accomplished by a high voltage and a potential divider. 
The measured voltage is proportional to the amount of electrons hitting the photo cathode. This signal can then be used for further measurements. Depending on the shunt there is always a trade-off between precision between the amount of signals and the energy of the initial signal. 
\subsection{Refraction index}\label{refrac index}


\newpage
\section{Experiment}
\subsection{Setup}\label{setup}
This experiment we are using a long and flat plastic scintillator to detect high energetic photons or charged particles like cosmic muons. The scintillator is connected with two photo multipliers (PMT) on each side, see figure \ref{fig:setup}.

\begin{figure}[H]
\centering
\begin{subfigure}[h]{0.4\textwidth}
\includegraphics[width=1\textwidth]{Plots/Scintillator.jpg}
\end{subfigure}
\begin{subfigure}[h]{0.59\textwidth}
\includegraphics[width=1\textwidth]{Plots/Raw.jpg}
\end{subfigure}
\caption{Pictures of the set up. Scintillator and PMTs on the right \cite{script}. Data acquisition rack on the left without cabling. Declaration see figure \ref{fig:cabling}.}
\label{fig:setup}
\end{figure}

Both of the PMT outputs are connected with the MSCF-16 unit, the yellow one in figure \ref{fig:setup} and \ref{fig:cabling}. The L labelled cable form the left PMT and the R labelled with an additional delay from the right one. Those Delay Boxes are just more cables the signal has to pass. From the MSCF-16 on the raw data will be delayed and passed to the analogue-to-digital converter (ADC) and time-to-digital converter (TDC) on the bottom of the pictures of the rack. And there it will be processed into the data we are working with later on. Another output leads over the third black cable to the Fan-In/-Out where the signal gets copied and passed to the Dual Gate Generator.

%%%%% Platz für mehr text :D # dont float mi floats 

\begin{figure}[H]
\centering
\includegraphics[width=1\textwidth]{Plots/Kabel2.jpg}
\caption{Data acquisition rack with cabling. Top left to right: MSCF-16, Quad Coincidence LeCroy 622 - not used, Dual Gate Generator LeCroy 222, Fan-In/-Out, NIM-ECL-NIM
module, two Delay Boxes.
Bottom: Gateway for PC connection, ADC, TDC.  }
\label{fig:cabling}
\end{figure}
%%%%%%%%%%%%%%%%%%%%%%%%%%%%%%%%%%%%% Kabel Schema aus Skript mit ins Bild paint'en?

The Gate Generators Delayed Output (DEL) produces a short signal for the TDC. This first signal starts the time measurement and gets stopped by the delayed signal of the MSCF-16. The length of the first triggering signal determines how long the second one will be measured. The ADC on the other hand receives a long signal from the gate generator. This leads to a good energy resolution but won't be necessary in this experiment.

%%% Description of how the data gets passed
\begin{figure}[H]
\centering
\includegraphics[width=1\textwidth]{Plots/Timing.png}
\caption{Schematic display on how the signals are passed between the components and what the TDC returns. \cite{script}}
\label{fig:timing}
\end{figure}

Now for a detailed description on what happens see figure \ref{fig:timing}. On top are the original signals the MSCF-16 receives. The signal from the right is delayed by the Delay Boxes. The first signal from the left PMT starts a coincidence gate. This is a variable time frame in which another signal has to appear to create a trigger which is used to start the time measurement. This helps to separate events and noise...
%%%%%%% WOZU ist denn dieses coincidence gate wirklich da?  
As explained above, the trigger signal is split in two at the Fan-In/-Out and processed in the gate generators for the special converter. In the TDC the delayed signal of the MSCF-16 then stops the time measurement. For us the time difference $t_R - t_L$ is of interest. This is proportional to the delay boxes delay and later on it will depend on where our radioactive sample is placed.

\subsection{Cosmic Muons} % aka Setup 2.0
% Cabling via NIM_ECL_NIM to the oszi
% no photos taken
% justierung des Spannungsverteilers

\begin{figure}[H]
\centering
\includegraphics[width=1\textwidth]{Plots/Spannung.jpg}
\caption{Picture of the voltage distribution for similar PMT signals. HV-L (red) supplies the left PMT and HV-R (black) the right one.}
\label{fig:voltage}
\end{figure}


\subsection{Time calibration}\label{time}
For this chapter we placed a $^{207}Bi$ source on top of the scintillator. A folding yardstick was fasten on top. For the calibration we placed the source at $100cm$ which is the middle of the scintillator. This should result in equal time delay, if the Delay Boxes are not used and the same cables are used. To determine at which time unit the most events occurred each $50\ 000$ events were measured to create a histogram. This treatment stays the same for the following calibration and the determination of the speed of light c.

Because we have no value how many time units are equal to $1ns$, the calibration has to be done. We started with an offset of $32ns$ on the delay boxes, to ensure that the left PMT always starts the coincidence gate for the sake of consistency. By increasing the additional delay the time difference will rise too and therefore a proportionality will be visible.

Each data set of $50\ 000$ events is presented in the histograms in the Appendix \ref{appendix}. Each is fitted by the Gaussian function, where $\mu$ is the expected value, $\sigma$ the standard derivation, A the maximum amount at $\mu$ and b a constant off-set. 
\begin{equation}
n(x) = A\cdot exp \left( -\frac{(x-\mu)^2}{2\sigma^2} \right) + b
\end{equation}

We are especially interested in the expected value $\mu$. The slope of this time difference and the external delay will tell us the relation between the units of time given by the TDC and $1ns$. This comparison with a linear fit is shown in figure \ref{fig:TimeCalibration}.

\begin{figure}[H]
\centering
\includegraphics[width=1\textwidth]{Plots/TimeCalibration.png}
\caption{Expected values versus external delay. Identification of how many units of time are equal to one $1ns$. }
\label{fig:TimeCalibration}
\end{figure}

This concludes in in the following conversion rule, with the real time T [ns], time difference t [units of time] and slope $a$ as shown in figure \ref{fig:TimeCalibration}:
\begin{equation}
T = \frac{t}{a}\:,\: dT=\sqrt{\left(\frac{dt}{a}\right)^2 + \left(\frac{t\cdot da}{a^2}\right)^2}
\end{equation}


\subsection{Speed of light measurement}\label{c determination}
The same procedure is now applied to determine the speed of light in the plastic scintillator material. Instead of changing the delay of the right PMT it remains constant with $32ns$ plus additional cables. We took 41 sets of $50\ 000$ events, the individuals are shown in Appendix \ref{appendix}. For each measurement the position of the $^{207} Bi$ sample was changed by $4cm$. The error on the position is estimated with $0.5cm$, due to the size of the sample itself was around $0.8cm$. An example on how the individual sets of events are displayed is shown in figure \ref{fig:histogram20}.

\begin{figure}[H]
\centering
\includegraphics[width=1\textwidth]{Plots/Pos/20cm.png}
\caption{Histogram of events at the position of 20cm. }
\label{fig:histogram20}
\end{figure}

Again the expected values are plotted but now against the positions of the sample. The plots with units of time is shown in Appendix \ref{appendix} in figure \ref{fig:c fit units}. With the conversion rule at the end of chapter \ref{time}, the values for T are calculated. The results after linear regression are shown in figure \ref{fig:c fit}.

\begin{figure}[H]
\centering
\includegraphics[width=1\textwidth]{Plots/PosTime.png}
\caption{Linear regression for the value of c. Converted time T against the samples position. }
\label{fig:c fit}
\end{figure}

Now by inverting the given slope we get our value for the speed of light in the plastic scintillator material. The error on this value will follow equation \ref{eq:c error}.
\begin{equation} \label{eq:c error}
c=\frac{1}{a}\:,\: dc=\frac{da}{a^2}
\end{equation}

To reference our value we searched for the refractive index $n$ of plastic scintillator material, due to the equation $ c= c_0 / n$. This describes the speed of light in a material, where $c_0 = 299\ 792\ 458 \frac{m}{s}$ is the speed of light in vacuum. We found a value of $n=1.58$ \cite{refractive index} and the following comparison.

\begin{table}[H]
\centering
\begin{tabular}{c|c|c}
Value & Fraction $[c_0]$ & Real value [$\frac{m}{s}$] \\ \hline \hline
Literature & $0.633$ & $189\ 742\ 062$  \\ \hline
Measurement & $0.1796 \pm 0.1486$ & $ 53\ 830\ 777 \pm 445\ 371$ \\ \hline
\end{tabular}
\end{table}

The literature value is more than three times larger than our measured value. This is not explainable by some small errors in positioning or the fit functions. Either the time calibration is very bad or another effect occurs in this experiment, which has not been discussed.

%%% das photon wird nicht straight zum PMT emittiert
%%% streueung an den rändern oder an anderen atomen
%%% längerer laufweg und damit gernigere Licht geschw
%%% UNSERE ANNAHME IST DIREKTER WEG!!!


%%% \section{Conclusion} oder so
%%% Fehlerquelle: Kabellängen 
%%% Conclusion oder was auch immer das sein soll ...


\section{Appendix}\label{appendix}
\subsection{Time calibration}
\begin{figure}[H]
\centering
\medskip
\begin{subfigure}{0.48\textwidth}
\includegraphics[width=\linewidth]{Plots/Time/0ns.png}
\end{subfigure}
\begin{subfigure}[c]{0.48\linewidth}
\includegraphics[width=\linewidth]{Plots/Time/4ns.png}
\end{subfigure}

\medskip
\begin{subfigure}{0.48\textwidth}
\includegraphics[width=\linewidth]{Plots/Time/8ns.png}
\end{subfigure}
\begin{subfigure}[c]{0.48\linewidth}
\includegraphics[width=\linewidth]{Plots/Time/12ns.png}
\end{subfigure}

\medskip
\begin{subfigure}{0.48\textwidth}
\includegraphics[width=\linewidth]{Plots/Time/20ns.png}
\end{subfigure}
\begin{subfigure}[c]{0.48\linewidth}
\includegraphics[width=\linewidth]{Plots/Time/28ns.png}
\end{subfigure}
\caption{Histograms for the time calibration in chapter \ref{time}. Part I }
\end{figure}

\begin{figure}[H]
\centering
\medskip
\begin{subfigure}{0.48\textwidth}
\includegraphics[width=\linewidth]{Plots/Time/40ns.png}
\end{subfigure}
\begin{subfigure}[c]{0.48\linewidth}
\includegraphics[width=\linewidth]{Plots/Time/52ns.png}
\end{subfigure}

\medskip
\begin{subfigure}{0.48\textwidth}
\includegraphics[width=\linewidth]{Plots/Time/60ns.png}
\end{subfigure}
\begin{subfigure}[c]{0.48\linewidth}
\includegraphics[width=\linewidth]{Plots/Time/70ns.png}
\end{subfigure}

\medskip
\begin{subfigure}{0.48\textwidth}
\includegraphics[width=\linewidth]{Plots/Time/80ns.png}
\end{subfigure}
\begin{subfigure}[c]{0.48\linewidth}
\includegraphics[width=\linewidth]{Plots/Time/95ns.png}
\end{subfigure}
\caption{Histograms for the time calibration in chapter \ref{time}. Part II }
\end{figure}


\subsection{Speed of light}
\begin{figure}[H]
\centering
\medskip
\begin{subfigure}{0.48\textwidth}
\includegraphics[width=\linewidth]{Plots/Pos/24cm.png}
\end{subfigure}
\begin{subfigure}[c]{0.48\linewidth}
\includegraphics[width=\linewidth]{Plots/Pos/28cm.png}
\end{subfigure}

\medskip
\begin{subfigure}{0.48\textwidth}
\includegraphics[width=\linewidth]{Plots/Pos/32cm.png}
\end{subfigure}
\begin{subfigure}[c]{0.48\linewidth}
\includegraphics[width=\linewidth]{Plots/Pos/36cm.png}
\end{subfigure}

\medskip
\begin{subfigure}{0.48\textwidth}
\includegraphics[width=\linewidth]{Plots/Pos/40cm.png}
\end{subfigure}
\begin{subfigure}[c]{0.48\linewidth}
\includegraphics[width=\linewidth]{Plots/Pos/44cm.png}
\end{subfigure}

\medskip
\begin{subfigure}{0.48\textwidth}
\includegraphics[width=\linewidth]{Plots/Pos/48cm.png}
\end{subfigure}
\begin{subfigure}[c]{0.48\linewidth}
\includegraphics[width=\linewidth]{Plots/Pos/52cm.png}
\end{subfigure}
\caption{Histograms for the c calculation in chapter \ref{c determination}. Part I }
\end{figure}

\begin{figure}[H]
\centering
\medskip
\begin{subfigure}{0.48\textwidth}
\includegraphics[width=\linewidth]{Plots/Pos/56cm.png}
\end{subfigure}
\begin{subfigure}[c]{0.48\linewidth}
\includegraphics[width=\linewidth]{Plots/Pos/60cm.png}
\end{subfigure}

\medskip
\begin{subfigure}{0.48\textwidth}
\includegraphics[width=\linewidth]{Plots/Pos/64cm.png}
\end{subfigure}
\begin{subfigure}[c]{0.48\linewidth}
\includegraphics[width=\linewidth]{Plots/Pos/68cm.png}
\end{subfigure}

\medskip
\begin{subfigure}{0.48\textwidth}
\includegraphics[width=\linewidth]{Plots/Pos/72cm.png}
\end{subfigure}
\begin{subfigure}[c]{0.48\linewidth}
\includegraphics[width=\linewidth]{Plots/Pos/76cm.png}
\end{subfigure}

\medskip
\begin{subfigure}{0.48\textwidth}
\includegraphics[width=\linewidth]{Plots/Pos/80cm.png}
\end{subfigure}
\begin{subfigure}[c]{0.48\linewidth}
\includegraphics[width=\linewidth]{Plots/Pos/84cm.png}
\end{subfigure}
\caption{Histograms for the c calculation in chapter \ref{c determination}. Part II }
\end{figure}

\begin{figure}[H]
\centering
\medskip
\begin{subfigure}{0.48\textwidth}
\includegraphics[width=\linewidth]{Plots/Pos/88cm.png}
\end{subfigure}
\begin{subfigure}[c]{0.48\linewidth}
\includegraphics[width=\linewidth]{Plots/Pos/92cm.png}
\end{subfigure}

\medskip
\begin{subfigure}{0.48\textwidth}
\includegraphics[width=\linewidth]{Plots/Pos/96cm.png}
\end{subfigure}
\begin{subfigure}[c]{0.48\linewidth}
\includegraphics[width=\linewidth]{Plots/Pos/100cm.png}
\end{subfigure}

\medskip
\begin{subfigure}{0.48\textwidth}
\includegraphics[width=\linewidth]{Plots/Pos/104cm.png}
\end{subfigure}
\begin{subfigure}[c]{0.48\linewidth}
\includegraphics[width=\linewidth]{Plots/Pos/108cm.png}
\end{subfigure}

\medskip
\begin{subfigure}{0.48\textwidth}
\includegraphics[width=\linewidth]{Plots/Pos/112cm.png}
\end{subfigure}
\begin{subfigure}[c]{0.48\linewidth}
\includegraphics[width=\linewidth]{Plots/Pos/116cm.png}
\end{subfigure}
\caption{Histograms for the c calculation in chapter \ref{c determination}. Part III }
\end{figure}

\begin{figure}[H]
\centering
\medskip
\begin{subfigure}{0.48\textwidth}
\includegraphics[width=\linewidth]{Plots/Pos/120cm.png}
\end{subfigure}
\begin{subfigure}[c]{0.48\linewidth}
\includegraphics[width=\linewidth]{Plots/Pos/124cm.png}
\end{subfigure}

\medskip
\begin{subfigure}{0.48\textwidth}
\includegraphics[width=\linewidth]{Plots/Pos/128cm.png}
\end{subfigure}
\begin{subfigure}[c]{0.48\linewidth}
\includegraphics[width=\linewidth]{Plots/Pos/132cm.png}
\end{subfigure}

\medskip
\begin{subfigure}{0.48\textwidth}
\includegraphics[width=\linewidth]{Plots/Pos/136cm.png}
\end{subfigure}
\begin{subfigure}[c]{0.48\linewidth}
\includegraphics[width=\linewidth]{Plots/Pos/140cm.png}
\end{subfigure}

\medskip
\begin{subfigure}{0.48\textwidth}
\includegraphics[width=\linewidth]{Plots/Pos/144cm.png}
\end{subfigure}
\begin{subfigure}[c]{0.48\linewidth}
\includegraphics[width=\linewidth]{Plots/Pos/148cm.png}
\end{subfigure}
\caption{Histograms for the time calibration in chapter \ref{time}. Part IV }
\end{figure}

\begin{figure}[H]
\centering
\medskip
\begin{subfigure}{0.48\textwidth}
\includegraphics[width=\linewidth]{Plots/Pos/152cm.png}
\end{subfigure}
\begin{subfigure}[c]{0.48\linewidth}
\includegraphics[width=\linewidth]{Plots/Pos/156cm.png}
\end{subfigure}

\medskip
\begin{subfigure}{0.48\textwidth}
\includegraphics[width=\linewidth]{Plots/Pos/160cm.png}
\end{subfigure}
\begin{subfigure}[c]{0.48\linewidth}
\includegraphics[width=\linewidth]{Plots/Pos/164cm.png}
\end{subfigure}

\medskip
\begin{subfigure}{0.48\textwidth}
\includegraphics[width=\linewidth]{Plots/Pos/168cm.png}
\end{subfigure}
\begin{subfigure}[c]{0.48\linewidth}
\includegraphics[width=\linewidth]{Plots/Pos/172cm.png}
\end{subfigure}

\medskip
\begin{subfigure}{0.48\textwidth}
\includegraphics[width=\linewidth]{Plots/Pos/176cm.png}
\end{subfigure}
\begin{subfigure}[c]{0.48\linewidth}
\includegraphics[width=\linewidth]{Plots/Pos/180cm.png}
\end{subfigure}
\caption{Histograms for the c calculation in chapter \ref{c determination}. Part V }
\end{figure}

\begin{figure}[H]
\centering
\includegraphics[width=1\textwidth]{Plots/PosTimeUnit.png}
\caption{Linear regression for the value of c. Original time units from the TDC against the samples position. }
\label{fig:c fit units}
\end{figure}

\newpage
\begin{thebibliography}{}

\bibitem{script} VMEScript.pdf, script for this experiment.

\bibitem{refractive index} \begin{verbatim}
https://eljentechnology.com/images/products/data_sheets/
  EJ-228_EJ-230.pdf
 https://www.crystals.saint-gobain.com/sites/imdf.crystals.com/
  files/documents/sgc-bc400-404-408-412-416-data-sheet.pdf
\end{verbatim} 


\end{thebibliography}
\end{document}


\documentclass[]{article}
\usepackage{graphicx}
\usepackage{hyperref}
\usepackage{amsmath}

%opening
\title{Vacuum Science and the Influence of Vacuum Conditions on Thin Film Growth}
\author{Gunther T\"urk, Jonas Lehnen}

\begin{document}

\maketitle
\tableofcontents
\begin{abstract}
In this experiment our main goal was to learn how one applies the kinetic gas theory at the example of a vacuum chamber. Especially how one is able to create high vacuum by turbo pumping the chamber and even further decrease the pressure by baking it. 

Write here, what we did and learned.
\end{abstract}

\section{Theorie}
\subsection{Kinetic Gas Theory}
In kinetic gas theory we assume a gas to consist of many microscopic spheres which don't interact with each other and take negligible volume in a container. The particles can move freely and spread homogeneously in a given container. When they crash onto the container walls they exert a force on these, we can measure as pressure. 
\[ p=F/A \]
It is measured in [Pa] or [bar].
\[ [Pa]=N/m^{2}     \qquad  1bar=10^{5}Pa		 \]
What we can measure as temperature is caused by the movement of the particles in the gas. So it has to be related to the kinetic energy, what explains why the pressure on the walls rises, if we increase the temperature. From experiments done in the 17th and 18th century we know the state equation for an ideal gas
\[ pV=nk_{B} T\]
 where $k_{B}\approx1,38J/K$ is the Boltzmann constant and temperature is measured in Kelvin.We assume for our purpose that all particles are incompressible hard spheres and that there are no frictional forces between the particles and the walls. Because the particle speeds are distributed isotropically there are $1/3N$ particles moving along each axis. For simplicity we consider a particle moving along the x-Axis with the velocity $v_{x}$. Initially it has the momentum $p_{x}=v_{x}*m$. After an elastic collision with a wall its momentum is $-p_{x}$, so the momentum change is $\Delta p_{x}=2p_{x}$. We can calculate the force on the wall by multiplying the momentum change of each collision with the ammount of collisions per time interval. In a box with surface A and length $ds=v*dt$ there are \[ N=\frac{N}{V}A *v* dt \] particles of which 1/6 is moving in x-direction. So the total momentum transfer is \[ \Delta p_{i}=\frac{2N* A* v* dt*m* v}{6V} \]. For the force on the wall we get \[ F=\frac{dp}{dt}=\frac{ANmv^{2}}{3V} \rightarrow p=\frac{Nmv^{2}}{3V} \Leftrightarrow p=\frac{N*E_{kin}}{3V}\]. Comparing this equation to the ideal gas law gives us that \[ v^{2}=\frac{3k_{b}T}{m_{p}} \] 
 Boltzmannstatistikk \\\\\\\\\\\\
 Assuming we have a canonical ensemble of gas molecules we can conclude that the distribution of particle speeds can be described by a Maxwell-Boltzmann-statistic. \[ f(v)=\frac{m^{\frac{3}{2}}}{2\pi k T} 4\pi v^{2} e^{\frac{-mv^{2}}{2kT}} \] Now we can calculate the expectation value for the speed $<v>$ and also $\sigma_{v}$
 \[<v>=\int_{0}^{\infty}vf(v)dv=\sqrt{\frac{8kT}{\pi m}}  \]
 \[ \sqrt{<v^{2}>}=\int_{0}^{\infty}v^{2}f(v)dv=\sqrt{\frac{3kT}{\pi m}}  \]
 The second result is consistent with the speed we derived earlier with simple thoughts about pressure. We can easily see that lighter particles move a lot faster than heavier ones. 
 \paragraph{Example}
 \mbox{}\\
 If we have an $H_{2}O$ molecule at room temperature (300K), we can calculate its thermal speed and energy.\[ 
 v=\sqrt{\frac{3kT}{\pi m}}=363\frac{m}{s} \]
 \[ E_{thermal}=\frac{1}{2}mv^{2}=1.09*10^{-22}J \]
\\\\\\\\
\subsubsection{Mean free path}\
The mean free path of a particle is described as the length after which $e^{-1}$ have not hit another particle yet. We can estimate the mean free path $\lambda$ in a gas if we know the total cross section $\sigma$  of a particle. The mean free path multiplied with the cross section gives us the effective volume the particle has covered. If in average a particle has then hit another particle the product of the covered volume with the particle density should be one.\[\lambda \sigma *\frac{N}{V}=1 \Leftrightarrow \lambda=\frac{V}{N\sigma}  \]
\paragraph{Example}mbox{}\\
We will calculate the mean free path of nitrogen (N2) molecules in atmospheric pressure and at $10^{9}mbar$
\\\\Hier muss noch das beispiel gerechnet werden.


\subsection{Desorption}
\subsection{Pumps}
We use different pumps to produce a high vacuum for the different pressure regimes. The pre pump is used to get to a low enough pressure to use the turbo pump which gets us to a very high vacuum.
\subsubsection{Pre Pump}
The pre pump (Fig.\ref{fig:prepump}) works by changing the volume in a pumping chamber while opening and closing 2 valves. During the inlet phase the valve towards the vacuum is open, and the one towards the atmosphere is closed, while the volume in the pump is increased. This is reversed during the outlet stage.

\begin{figure}
	
	\centering
	\includegraphics[width=0.7\linewidth]{Bilder/PrePump}
	\label{fig:prepump}
	\caption{Shematic representation of a pump}
	
\end{figure}


The pressure limit we can reach with the pre pump alone is on the one hand caused by leaks of the valves and the piston. On the other hand there is a pressure limit even with perfect valves by the air ($V_{0} $) at atmospheric pressure which remains in the cylinder at the outlet stage . With isotherm expansion in the cylinder we can calculate this minimum pressure to be: \[ p_{min}=p_{Atmosph.}\frac{V_{0}}{V_{max}} \]
\subsubsection{Turbo Pump}
Different from the pre pump which works by producing a pressure gradient between two chambers, a turbo pump works through the principle of momentum transfer. Atoms in a chamber don't bounce elastically of the walls but stick to them through adsorption for a short amount of time, so it is possible to accelerate them with spinning rotors in the turbo pump and push the atoms to the outlet. This only works if the mean free path of a particle is lager than the distance between the rotor blades, which is why we connect the outlet of the turbo, to the pre pump. Our turbo pump is able to evacuate 300 litres of gas per minute.
\subsection{Sputtering}

\subsection{Flow}
\subsubsection{Laminar}
\subsubsection{Turbulent}
\subsubsection{Resistance and Conductivity}
\subsection{Gauges}
There are 3 different kinds of pressure gauges in our vacuum chamber, each designed to work best at different pressures. While it is possible to change technical parameters for every gauge the mode of operation of each gauge suits it best for a specific regime.
\subsubsection[Capacitive]{Capacitative Pressure Gauge}
The capacitive pressure gauge is basically a plate capacitor with a insulator in between. One plate is connected to the chamber, so that the pressure difference will push the plates together. That causes the conductivity to increase what we then can measure. It can be used at higher pressures. 
\subsubsection[Pirani]{Pirani Pressure Gauge}
A pirani gauge consists of a metal filament (usually platinum) connected to an electrical circuit. The filament is heated when the current flows through. At every collision with a gas molecule it gives heat to the molecule, what causes the temperature of the filament to drop. The rate of molecules hitting the filament is proportional to the amount of molecules in the chamber, so cooling of the filament slows down with sinking pressures. Because the resistance of the filament rises with temperature we can use the measurement of it to get a pressure reading from our gauge. It may be used at pressures between $0.5\ mbar$ to $10^{-3}mbar$
\subsubsection[Penning]{Penning Pressure Gauge}
Bild von nem penning pressure gauge
\\\\\\\\\
A penning gauge is made out of a cyllindric metal cage, serving as the cathode and a inner filament which serves as the anode. In a right angle to the plane of the field lines is a magnetic field, causing electrons emitted from the cathode to move in long spiral orbits towards the anode. In this process they ionise the molecules left in the chamber what causes cascades of electrons emitted to the anode, even an very low pressures. The current decreases with lower pressure, since frequency of collisions as well as cascade intensity both decrease. It can be used for pressures between $10^{-2}mbar$ to $10^{-7}mbar$.

\subsection{Questions}
\subsubsection{Day1}
\begin{itemize}  
	\item What causes the limit on the pressure one can reach with the pre-pump alone? $\surd$
	\item  Calculate the mean free path of a nitrogen (N2) molecule in atmospheric pressure and at $1 \cdot 10^{−9}$ mbar 
	\item Based on the data from section 5.3, how much gas is desorped from the walls after the turbo reaches full speed? Hint: Consider the data over the time period after the turbo reaches speed. This is the pressure drop in a fixed time. The volume of the chamber is 30 litres and you can assume the chamber is at a constant temperature of 293K. The mass of air is approximately 29g/mol.
	 

	\item What is the purpose of baking out a chamber? 
	\item What is the thermal energy at room temperature and the corresponding thermal speed of a water molecule (H2O)?
\end{itemize}
\subsubsection{Day2}
\begin{itemize}
	 \item Can you name some properties which might arise due to scaling down the thickness of a sample? \item How have the elemental peaks changed from the spectra of the first day? What can you say about the baking process? How will the baking process affect the quality of the samples being grown? \item Why does the pressure oscillate with clear periods of increasing and decreasing pressures during baking? \item What happens to the plasma as you increase the power being applied to the cathode?
	Page 14 of 34
	\item Why do we pre-sputter the target before deposition? \item  Plot the Polanyi-Wigner equation as a function of desorption energy for an ensemble of $5.35 \cdot 10^{13}$ particles for $T_w$ = 300K, 500K, 800K, 1000K for Edes between 0 and 60 kJ/mol (universal gas constant 8.31 J/mol). Scale the y-axis to $10^{13}$ as a maximum.
	
\end{itemize}
\section{Experiment}
In the experiment we were allowed to operate a vacuum chamber with a volume of 30 litres. The inside is reachable by the main window which is hold by eight bolts and a copper seal. The sputtering targets are at the bottom while a rotatable disk, where the sample was placed, is in the center. This one is operate-able from the outside. At the backside the pressure gauges are connected. 
The chamber itself was connected to a turbo pump, which itself was connected to a valve with electric controls as well a brass cap for the venting. Behind the turbo pump the pre-pump with a 25mm diameter tubing was placed. Above the pumping systems is the mass spectrometer with its filament.

\subsection{Description}
\subsection{Data Analysis}
\subsection{Evaluation}

\end{document}


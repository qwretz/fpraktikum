\documentclass[]{article}

%opening
\title{Vacuum Science and the Influence of Vacuum Conditions on Thin Film Growth}
\author{Gunther Türk, Jonas Lehnen}

\begin{document}

\maketitle
\tableofcontents
\begin{abstract}
Write here, what we did and learned.
\end{abstract}

\section{Theorie}
\subsection{Kinetic Gas Theory}
In kinetic gas theory we assume a gas to consist of many microscopic particles which don't interact with each other and take no or negligible volume in a container. The particles can move freely and spread homogeneously in a given container. When they crash onto the container walls they exert a force on these, we can measure as pressure. 
\[ p=F/A \]
It is measured in [Pa] or [bar].
\[ [Pa]=N/m^{2}     \qquad  1bar=10^{5}Pa		 \]
What we can measure as temperature is caused by the movement of the particles in the gas. So it has to be related to the kinetic energy, what explains why the pressure on the walls rises, if we increase the temperature. From experiments done in the 17th and 18th century we know the state equation for an ideal gas
\[ pV=nk_{B} T\]
 where $k_{B}\approx1,38J/K$ is the Boltzmann constant and temperature is measured in Kelvin. 


\subsection{Pumps}
\subsubsection{Pre Pump}
\subsubsection{Turbo Pump}
\subsection{Sputtering}
\subsection{Flow}
\subsubsection{Laminar}
\subsubsection{Turbulent}
\subsubsection{Resistance}
\subsection{Gauges}
\subsection{Questions}
\subsubsection{Day1}
\begin{itemize}  
	\item What causes the limit on the pressure one can reach with the pre-pump alone? 
	\item  Calculate the mean free path of a nitrogen (N2) molecule in atmospheric pressure and at 1 × 10−9 mbar 
	\item Based on the data from section 5.3, how much gas is desorped from the walls after the turbo reaches full speed? Hint: Consider the data over the time period after the turbo reaches speed. This is the pressure drop in a fixed time. The volume of the chamber is 30 litres and you can assume the chamber is at a constant temperature of 293K. The mass of air is approximately 29g/mol.
	 

	\item What is the purpose of baking out a chamber? 
	\item What is the thermal energy at room temperature and the corresponding thermal speed of a water molecule (H2O)?
\end{itemize}
\subsubsection{Day2}
\begin{itemize}
	 \item Can you name some properties which might arise due to scaling down the thickness of a sample? \item How have the elemental peaks changed from the spectra of the first day? What can you say about the baking process? How will the baking process affect the quality of the samples being grown? \item Why does the pressure oscillate with clear periods of increasing and decreasing pressures during baking? \item What happens to the plasma as you increase the power being applied to the cathode?
	Page 14 of 34
	\item Why do we pre-sputter the target before deposition? \item  Plot the Polanyi-Wigner equation as a function of desorption energy for an ensemble of 5.35x1013 particles for Tw = 300K, 500K, 800K, 1000K for Edes between 0 and 60 kJ/mol-1 (universal gas constant 8.31 J/mol-1). Scale the y-axis to 1013 as a maximum.
	
\end{itemize}
\section{Experiment}
\subsection{Description}
\subsection{Data Analysis}
\subsection{Evaluation}

\end{document}


\documentclass[]{article}
\usepackage{graphicx}
\usepackage{hyperref}
\usepackage{amsmath}
\usepackage{caption}
\usepackage{subcaption}
\usepackage{ngerman}
\usepackage[utf8]{inputenc}
\usepackage{float}

%opening
\title{Hochauflösende $\gamma$ Spektroskopie}
\author{Gunther T\"urk, Jonas Lehnen}

\begin{document}
	
	\maketitle
	\begin{abstract}
		In diesem Experiment wollen wir verstehen wie der $\gamma$ Zerfall gemessen wird um die genaue Energie des Photons zu bestimmen. Nach dem ein Elektron gefangen oder ein $\beta$ Zerfall statt fand, ist der Atomkern angeregt. Die Abregung des Kerns erfolgt über eben dieses Photon, welches ein Germanium Detektor messen kann. In diesem Zuge werden wir besprechen in wie weit Elektronik und Hintergrundstrahlung dieses Experiment beeinflussen.
	\end{abstract}
	
	\tableofcontents
	
	
	\newpage
	\section{Theorie}
	\subsection{Radioaktive Strahlung}
	Grundsätzlich existieren drei Arten des radioaktiven Zerfalls. Beginnend mit dem $\alpha$ Zerfall, bei dem ein ${He}\ ^{2+}$ Teilchen aus einem größeren Atomkern emittiert wird. Die genauen Zerfallskanäle sind in Tabelle \ref{tab:zerfall} dargestellt. $\alpha$ Strahlung selbst ist leicht abgeschirmt, so besteht bereits nach $10cm$ Luftweg keine Belastung für Menschen. Probleme entstehen erst bei Alphastrahlern im Körper, da dann die gesamte Energie deponiert wird und unter anderem Krebs auslösen kann.
	
	\renewcommand{\arraystretch}{2}
	\begin{table}[H]
		\centering
		\begin{tabular}{c||c}
			Zerfallsart & Formel   \\ \hline
			$\alpha$ & $_Z^A X \rightarrow\ _2^4\alpha + _{Z-2}^{A-4} Y$ \\ \hline
			$\beta^+$ &   $_Z^A X \rightarrow\  _{Z-1}^{A}Y + e^+ + \nu_e$ \\ \hline
			$\beta^-$ &   $_Z^A X \rightarrow\  _{Z+1}^{A}Y + e^- +  \bar{\nu}_e$ \\ \hline
			$\gamma$ &  $_Z^A X^* \rightarrow\ _Z^A X + \gamma$  \\ \hline
			$EC$ &   $_Z^A X  + e^- \rightarrow\ _{Z-1}^{A}Y + \nu_e $
		\end{tabular}
		\caption{Darstellung aller möglichen radiokativen Zerfälle. Die Kernendprodukte können sich, bis auf beim $\gamma$ Zerfall, noch in einem angeregten Zustand befinden. Dies wird durch $X^*\ bzw.\ Y^*$ dargestellt.  }
		\label{tab:zerfall}
	\end{table}
	\renewcommand{\arraystretch}{1}
	
	Wie in jedem Zerfall gilt auch für die $\beta$ Zerfälle, dass die Masse des ursprünglichen Kerns größer ist als die der Zerfallsprodukte. Dies liegt an der Energieerhaltung, da zusätzliche kinetische Energie entsteht um das Zerfallsteilchen fort zubewegen. Hier unterscheidet man je nach Ladung von Elektron oder Positron zwischen negativen um positiven $\beta$ Zerfall. Damit auch die Flavourerhaltung der Leptonen berücksichtigt wird, entstehen hierbei zusätzliche Neutrinos. Im Kern selbst zerfällt dabei ein Neutron bzw. Proton zu genau dem anderen Kernteilchen. Der Zerfall eines freien Protons ist theoretisch möglich jedoch wird die Halbwertszeit auf $10^{35}$ Jahre geschätzt. Unser Universum ist bisher erst ca. $10^{10}$ Jahre alt. Es ist also sehr unwahrscheinlich einen solchen Nachweis in einem Menschenleben erbringen zu können.
	
	Alle genannten Strahlungsarten können nun in einem angeregten Kernzustand enden. An dieser Stelle setzt nun der $\gamma$ Zerfall ein. Diese angeregten Zustände sind durch Kernspin und Parität charakterisiert. Dieser Zerfall entspricht einer Umstrukturierung des Kernaufbaus und die dabei gewonnene Energie hat zwei Möglichkeiten umgesetzt zu werden. Entweder als Photon des $\gamma$ Zerfalls oder auch direkt an ein Elektron, welches dann das Atom verlassen kann, sofern die Bindungsenergie aufgebracht werden konnte. Dieser Prozess wird Innere Umwandlung genannt und kann nur mit Elektronen einer s-Schale stattfinden, da nur diese eine Aufenthaltswahrscheinlichkeit im Kern besitzen.
	
	\begin{figure}[H]
		\centering
		\begin{subfigure}[b]{.48\textwidth}
			\centering
			\includegraphics[width=1\textwidth]{Plots/Cobalt1.png}
			\caption{Schema der Energieniveaus und Zerfallsübergänge. Vor dem eigentlichen $\gamma$ Zerfalls findet ein $\beta^-$ Zerfall statt. Am rechten Rand sind Kernspin und Parität vermerkt. Die Halbwertszeit ist mit mit 5.272 Jahren ebenfalls dargestellt.}
		\end{subfigure}
		\begin{subfigure}[b]{.48\textwidth}
			\centering
			\includegraphics[width=1\textwidth]{Plots/Cobalt2.png}
			\caption{Spektrum des $\gamma$ Zerfalls. Beide Peaks entsprechen den im $^{60}Ni$ angegebenen Energien des Übergangs. }
		\end{subfigure}
	
	\end{figure}
	
	Zusätzlich existiert noch der Gegensatz zum $\beta$ Zerfall. Beim Electron Capture EC wird ein Elektron eingefangen statt ausgesendet. Durch die Änderung des Kerns befindet sich jetzt das gesamte Atom in einem angeregten Zustand, statt der Kern. Am wahrscheinlichsten ist, dass ein $e^-$ aus der K-Schale gefangen wird. Dabei entsteht ein freier Platz auf einem niedrigen Niveau, welcher dann von oben wieder nach besetzt wird. Dabei entsteht nun Röntgenstrahlung, statt $\gamma$ Strahlung. Der Unterschied zwischen besteht also darin, ob das Photon im Kern ($\gamma$) oder aus der Schale (Röntgen) stammt.
	
	
	\subsection{Photonische Wechselwirkung mit Materie}
	Je nach Energiebetrag eines Photons gibt es verschiedene Arte um mit Materie zu interagieren. Hauptsächlich geschieht die Wechselwirkung durch den Compton-Effekt. Dabei wird das Photon elastisch an meist einem Elektron gestreut, prinzipiell sind aber alle geladenen Teilchen denkbar. Das Elektron erhält zusätzliche kinetische Energie und die Wellenlänge des Photons erhöht sich. Dieser Effekt tritt bei fast allen Photonenenergien auf. Seine Wahrscheinlichkeit kann jedoch in bestimmten Regionen durch andere Effekte unterdrückt sein.
	
	Der Photoeffekt ist einer davon. Er dominiert für geringere Energien bis ca. $500\ keV$. Bei diesem Prozess handelt es sich um die komplette Absorption von Energie und Impuls des Photons durch ein Elektron, welches dabei die Möglichkeit erhalten kann die Bindungsenergie zu überwinden. Überschüssige Energie findet sich im Impuls wieder.
	
	Der andere Effekt, der für hohe Energien ab einigen $MeV$ dominiert, ist die Paarbildung. Hierbei besitzt ein Photon genug Energie um ein Elektron und ein Positron zu erzeugen. Da ein Photon sich in jedem Bezugssystem, also auch im Ruhesystem von $e^-,e^+$, mit Lichtgeschwindigkeit c bewegt, wird hierbei ein Kern benötigt um die Impulserhaltung zu gewährleisten. 
	
	Alle diese Effekte führen dazu, dass sich meist ein Elektron mit überschüssiger Energie in der Materie bewegt. Dabei kann nun Bremsstrahlung auftreten. Es werden somit wieder Photonen frei und eine Kette aus diesem Prozessen findet statt. Dies ist der Grund warum im folgenden Kapitel überhaupt die Möglichkeit besteht radioaktive Zerfälle elektrisch wahrzunehmen.
	
	\begin{figure}[H]
		\centering
		\includegraphics[width=.7\textwidth]{Plots/PhotonInteraktion.png}
		\caption{Darstellung welche Interaktion der Photonen mit Materie in welchen Energiebereichen am wahrscheinlichsten ist. }
	\end{figure}
	
	\subsection{Detektion von Strahlung}
	Es gibt nun je nach Verwendungszweck verschiedene Arten um Teilchen nachzuweisen, insbesondere auch um radioaktive Strahlung zu detektieren. Für die meisten Nachweise werden Szintillatoren verwendet. Das Prinzip besteht darin, dass das Strahlungsteilchen seine Energie an das Material des Szintillator abgibt und dabei Elektronen frei werden. Dies kann durch Streuprozesse als auch den Photoeffekt geschehen. Dieser spielt im zweiten Schritt eine große Rolle, da die freien Elektronen meist noch genug Energie besitzen um Bremsstrahlung zu emittieren. Durch die Wiederholung des Prozesses entsteht ein Lichtblitz der sich durch das Material fortsetzt. 
	
	Zur digitalen Auswertung des Ereignisses wird dann ein Photomultiplier benötigt. Dieser beruht ebenfalls auf dem Photoeffekt, doch nun wird das frei Elektron von einer Anode angezogen, da diese auf einem höheren Potential liegt als die Anode, welche vom Photon getroffen wurde. Mit der aus dem elektrischen Feld gewonnenen Energie können nun an der Anode weitere Elektronen herausgeschlagen werden. Durch stets versetzte Anoden, damit die Elektronen keine Anode überspringen, nimmt deren Anzahl pro Anode exponentiell zu. Zum Schluss wird noch der Stromfluss gemessen, welcher proportional zur anfangs deponierten Energie ist. Diese Anwendung ist gut solange die genaue Energie eines Ereignisses nicht sonderlich wichtig ist sondern nur dessen Vorkommen und Aktivität. Sie sind jedoch besser als ein Geiger-Müller Zählrohr, bei dem nur die Anzahl der Teilchen messbar ist.
	
	Um eine genaue Energieauflösungen zu erhalten, welche den Sinn der hochauflösenden $\gamma$ Spektroskopie darstellen, sollte man nun Halbleiterdetektoren benutzen. Hierbei 
	werden statt Szintillatoren Halbleiterbauteile benutzt. In diesem Experiment wird wie auch meist ein koaxialer Germanium Detektor verwendet. Das Prinzip dahinter ist die Verwendung der Sperrschicht, welche entsteht wenn verscheiden dotierte Versionen eines Halbleiters aneinander geführt werden. Der Überschuss von Elektronen im n-dotierten Material wird vom Mangel an $e^-$, bzw. Überschuss an Löcher, abgesaugt. Es entsteht ein Bereich indem nur noch die feste Ladungsträger der Gitterstruktur vorhanden sind und dadurch ein elektrisches Feld. Durch anlegen einer Spannung in Sperrrichtung kann man weitere Elektronen bzw. Löcher von dieser Sperrschicht absaugen. Der positive Pol der Spannungsquelle liegt also an der n-dotierten Seite an. Mit hohen Spannungen kann man nun den ganzen Halbleiter in eine solche Sperrschicht umwandeln.
	
	\begin{figure}[H]
		\centering
		\includegraphics[width=.66\textwidth]{Plots/Energie.png}
		\caption{Vergleich der Energieauflösungen von Szintillatoren und Halbleiter Detektoren. Man erkennt sehr gut, bei Halbleitern die Peaks schärfer sind und damit nahe Peaks deutlich unterscheidbar sind.}
		\label{fig:energiaufloesung}
	\end{figure}
	
	Tritt nun ein Photon in diesen Detektor ein, so entsteht ein Elektron und gleichzeitg ein Loch. Elektronen können nach vorigem Kapitel nun weitere erzeugen. Durch das starke elektrische Feld werden die freien Ladungsträger nun schnell abgesaugt. Der entstandene Strom ist nun wieder direkt zu Photonenenergie proportional. Anders als zum Szintillator ist die Energieauflösunge der Germaniumdetektors größer. Dies liegt an der benötigeten Energie um Elektronen im Material zuerzeugen. Während diese im Germanium Halbleiter bei ca. $0.7eV$ liegt, benötigt man in einem NaI Szintillator fast $3eV$. \cite{energieaufloesung} Es geht viel Energie auch in Gitterschwingungen ein, welche dann nicht mehr vom PMT gemessen werden kann. Die Photonenenergie erzeugt im Szintillator nun weniger Elektronen und damit weniger Strom. Dadurch ist auch die Energieauflösung schlechter, wie in Abbildung \ref{fig:energiaufloesung} dargestellt. Es wurde die selbe radioaktive Quelle untersucht. Oben ist das Spektrum der Szintillators dargestellt, während darunter das Spektrum des Halbleiter Detektors liegt. Man erkennt sehr gut, dass einzelne Peaks oben eigentlich aus vielen einzelnen bestehen. Daher ist die Verwendung von, wie im Folgenden auch, eines Germanium Detektors sinnvoll.
	
	Großer Nachteil, welcher in der Auswertung eine Rolle spielen wird, ist die benötigte ständige Kühlung. Damit der Germanium Detektor wie erwartet funktionieren kann benötigte er eine Temperatur von ca. $T=77K$. Dies wird mit flüssigem Stickstoff umgesetzt. Der Grund für die Wichtigkeit dieser Temperatur ist, wie auch der große Vorteil, die geringe Bandlücke des Halbleiters von $E_g=0.7eV$. Bereits bei Zimmertemperatur ist die thermische Energie im Kristall groß genug um so Elektronen Loch Paare zu erzeugen. Dieser Stromfluss wird als Rauschen wahrgenommen und überdeckt das eigentliche Spektrum der zu messenden radioaktiven Strahlung.
	
	
	\subsection{Aktivität}
	Zur vollständigen Behandlung von radioaktiver Strahlung muss man nun auch das prinzipielle Zerfallsgesetz erwähnen. Ausgehend von einer Zerfallskonstante $\lambda$, welche die Wahrscheinlichkeit des Zerfalls eines Atomkerns bezeichnet, so gelten folgende Gesetzte:
	
	\begin{equation}
	\frac{dN(t)}{dt} = - \lambda N(t) \: \rightarrow \: N(t)=N_0 \cdot e^{\lambda t} = N_0 \cdot e^{t/\tau}
	\label{eq:zerfallsgesetz}
	\end{equation}
	Die Lebensdauer $ \tau = 1/\lambda $ beschreibt in diesem Zusammenhang zu welcher Zeit nur noch $1/e= 36.8\% $ übrig sind. Einfach für die Handhabung ist die Halbwertszeit $T_{1/2}$ bei welcher die Hälfte zerfallen ist. Dabei gilt der simple Zusammenhang $ T_{1/2} = \tau \cdot ln(2) $. 
	
	Als Aktivität wird dabei bezeichnet wie stark ein Stoff radioaktive Strahlung emittiert. SIe ist damit abhängig wie groß die Wahrscheinlichkeit des Kernzerfalls ist und wie viel noch zu einer gegebenen Zeit übrig ist. Demnach gilt:
	
	\begin{equation}
	A(t) = \lambda \cdot N(t) = \frac{ln(2)}{T_{1/2}} \cdot N(t)
	\label{eq:activity}
	\end{equation}
	
	
	\newpage
	\section{Experiment}
	\subsection{Setup}
	Wie in der Theorie bereits angesprochen wird für dieses Experiment ein mit flüssigem Stickstoff gekühlter Koaxial Germanium Detektor verwendet. Es handelt sich hierbei um eine schwach n-dotierten Germanium Zylinder mit Bohrloch. Die Wände des Bohrloches werden mit stark p-dotiertem Germanium beschichtet. Dabei entsteht bereits eine Sperrschicht, welche durch das anlegen einer Spannung von bis zu $U_0=3kV$ auf den ganzen Zylinder ausgeweitet wird. Der grobe Aufbau ist in Abbildung \ref{fig:setup} dargestellt.
	
	\begin{figure}[H]
		\centering
		\includegraphics[width=.8\textwidth]{Plots/Setup.png}
		\caption{Versuchsaufbau: 1) Radioaktive Quelle, 2) Fenster zur Vakuumkammer, 3) Germanium Detektor, 4) Preamplifier, 5) Hochspannungsquelle, 6) MCA inkl. ADC, 7) Main Amplifier, 8) Pulser, 9) Teilchensieb, 10) Kupferstab zur Kühlung, 11) Flüssiges $N_2$  }
		\label{fig:setup}
	\end{figure}
	
	Die Signale werden dann vom Preamplifier verstärkt. Da es sich bei einem solchen Signal im Optimalfall um eine Delta Funktion handelt, bei der die Amplitude direkt zur Energie proportional ist. Dies kann nur schlecht elektronisch realisiert werden, daher besitzt das Ausgangssignal hier noch einen abklingenden Teil, siehe Abbildung \ref{fig:signals}. Dieser hat keine zusätzliche Information über die Energie der Strahlung sondern entsteht, da sich der Schaltkreis bzw. Operationsverstärker nicht instantan wieder entladen kann. Es entsteht hierbei also eine kurze Totzeit in welcher der Preamplifier nicht arbeiten kann. 
	
	\begin{figure}[H]
		\centering 
		\includegraphics[width=1\textwidth]{Plots/deadtime.png}
		\caption{Darstellung der Signalübertragung zwischen Preamplifier, Main Amplifier und ADC Output an den Computer. \cite{signalverarbeitung}  
		\label{fig:signals}
	\end{figure}
	
	Danach wird das Signal vom Main Amplifier umgeformt, damit der ADC dieses später in ein digitales Signal umwandeln kann. Die wichtigste Bedingung daran ist eine gemeinsame Grundlinie. Klingt das Signal wegen langsamer Kondensatorentladung nicht auf eine konstante 0 ab, so kann nicht genau bestimmt werden wann das Signal zu Ende ist. Ebenso ist es ungünstig Delta Funktionen zu messen, da das Zeitfenster zur Energiebestimmung sehr klein ist. Daher werden aus diesen Dreieck-Signalen von Main Amplifier Gausspeaks geformt. Es entspricht einer Schaltung eines Hochpasses mit folgendem Tiefpass. Der Hochpass erstellt die 0 Linie während der Tiefpass die Spitzen glättet. Weiterhin bleibt die Energieinformation im Maximum des Peaks.
	
	Mit der Breite des Gauss kann man nun bestimmen wie schnell die Messung sein kann. Für kleine Zeitkonstanten TC, die Breite der Gausskurve entspricht $7.3\cdot TC$ \cite{signalverarbeitung}, ist der Peak nun schmal. Damit besteht die Möglichkeit mehr Events einzeln zu messen. Die Energieauflösung nimmt jedoch wegen schmalerem Signal wiederum ab. Eine Möglichkeit dies zu verhindern ist die Probe weiter vom Detektor zu platzieren, wie später noch beschrieben wird.
	
	Es gibt nun wie in Abbildung \ref{fig:signals} gezeigt, die Möglichkeit zwei Pulse, die zu nah aneinander sind abzulehnen. Dazu benötigt man ein Ablehnungssignal (Reject Signal), welches von einem extra Modul des Main Amplifiers ausgesendet wird. Dieser Pileup Rejector entscheidet also ob eine gültige Messung vorliegt. Dies führt jedoch zu einer größeren Totzeit des Detektors und damit zu einer schlechten Bestimmung der Aktivität einer Quelle. In unserem Fall wird dieser Mechanismus bewusst nicht benutzt um eine Energiekalibration mit den $^{60}Co$ Peaks durchzuführen, aber auch da es für geringe Zählrate unnötig ist.  
	%%%%%%%%%% REF 60Co -- Plots von Jonas
	Die Aktivität der zu vermessenden Präparate, siehe Tabelle \ref{tab:T 1/2}, ist mit großer Wahrscheinlichkeit bereits so gering, dass die empfohlenen $25cm$ Abstand zum Detektor, nicht mehr notwendig sind. Angeblich seien diese von 1995 und damit sind alle hoch aktiven Quellen wie $^{60}Co$ und $^{22}Na$ bereits auf $2^{-20a/T_{1/2}} \approx 1/16 \ bzw. \ 1/1024 $ der ursprünglichen Menge und damit Aktivität zerfallen. Alle Materialien anderen besitzen wegen hoher Halbwertszeit sowieso nur geringe Aktivitäten. 
	Daher wurden die Präparate direkt an der Vakuumkammer fixiert.
	
	\begin{table}[H]
		\centering
		\begin{tabular}{c|c|c|c|c|c|c}
			Präparat & $^{60}Co$& $^{133}Ba$& $^{152}Eu$& $^{137}Cs$& $^{22}Na$ & $^{241}Am$ \\ \hline 
			Halbwertszeit $T_{1/2} [a]$ & 5.27 & 10.51 & 13.52 & 30.17 & 2.60 & 241.06  \\ 
		\end{tabular}
		\caption{Verwendete radioaktive Materialien und deren Halbwertszeit.}
		\label{tab:T 1/2}
	\end{table}
	
	Zuletzt wird das Signal von Analog-To-Digital Converter (ADC) umgeformt, damit der aus dem Peak ein langes Signal wird, welches dann die Energieinformation beinhaltet. Dazu wird prinzipiell die Energie in eine Kondensator gespeichert und die Zeit der Entladung gemessen. Diese wird als Signal an den PC weitergegeben und entspricht weiterhin der Photonenenergie. Ein positives Ablehnungssignal würde die Weitergabe des Signals verhindern. Effektiv wäre dies dann keine Messung.
	
	\subsection{Durchführung}
	Da der Detektor zu warm war um den Versuch durchzuführen mussten wir leider auf Altdaten einer zurückgreifen. Wir haben dabei für die Daten der Gruppe 23, Sommersemester 2018, entschieden, da diese den Pulser bei jeder Messung eingeschaltet haben, was eine exakte Totzeitkorrektur für jede Messung ermöglicht. Da die Durchführung des Versuchs darin besteht verschiedene Proben auf den Detektor zu legen und eine Messung am PC zu starten gehen wir davon aus, dass hier keine relevanten experimentellen Fehler gemacht wurden. 
	%%% Signal verarbietung? JA!
	\subsection{Auswertung}
	Das Ziel des Versuchs ist es die Aktivität und die Energie Gamma-Peaks der Verschieden Proben zu bestimmen. Dazu führen wir als erstes eine Totzeitkorrektur aus gefolgt von einer Korrektur der Hintergrundstrahlung. Mit den korrigierten Daten können wir dann eine Energiekalibrierung anhand bekannter Peakenergien durchführen. Nachdem alles Kalibriert ist müssen wir nur noch die gemessenen Peaks fitten und können die Proben identifizieren, bzw. die gemessenen Werte mit Literaturwerten vergleichen. 
	\subsubsection{Totzeitkorrektur} 
	Der Pulser ist konstant auf eine Energie am oberen Ende des Spektrums unseres Detektors und eine Frequenz von 50Hz eingestellt (\ref{fig:rawNa22}). Diese Frequenz ist so niedrig, dass sie unsere Totzeit nur minimal erhöht und gleichzeitig eine relativ präzise Bestimmung der Totzeit zulässt. Die Totzeitkorrektur berechnet sich wie folgt:
	\begin{equation}
	k_{tot}=\frac{Gemessene \: Ereignisse}{t*50Hz}
	\end{equation}
	\begin{figure}
		\centering
		\includegraphics[width=1\linewidth]{RawPlots/Na22}
		\caption{The large peak at channel 13760 is the peak of the Pulser}
		\label{fig:rawNa22}
	\end{figure}
	\begin{figure}
		\centering
		\includegraphics[width=0.7\linewidth]{Plots/Totzeit/Na22}
		\caption{Example Fit of the Pulser Peak}
		\label{fig:DeadTimena22}
	\end{figure}
	
	\begin{table}[H]
		\centering
		\begin{tabular}{|c|c|c|c|}
			\hline
			Messung & Relative Totzeit [\%] & korrigierte Messzeit [s] & Messzeit [s] \\ \hline\hline
			hintergrund & 99.83 & 3593.92 & 3600 \\ \hline
			Na22 & 99.84 & 898.52 & 900 \\ \hline
			Totzeit & 99.85 & 299.56 & 900 \\ \hline
			unbekannt & 99.86 & 898.75 & 900 \\ \hline
			Eu152 & 95.03 & 855.26 & 900 \\ \hline
			Cs137 & 96.90 & 872.14 & 900 \\ \hline
			Co60 & 99.37 & 894.31 & 900 \\ \hline
			Ba133 & 96.87 & 871.84 & 900 \\ \hline
			Aufgabe5 & 93.26 & 839.33 & 900 \\ \hline
			Am241 & 97.43 & 876.88 & 900 \\ \hline
			\hline
		\end{tabular}
		\caption{Die Gemessenen Ereignisse ergeben sich aus dem Integral über den Gaußpeak mit der Energie des Pulsers.\ref{fig:DeadTimena22} \label{}}
	\end{table}
	
	
	
	\subsubsection{Korrektur der Hintergrundstrahlung}
	Um nur die Peaks der Proben zu erhalten müssen wir die Hintergrundstrahlung die z.B. aus den Wänden des Raums kommt korrigieren. Hierzu nutzen wir die Hintergrundmessung. Wir modellieren das Strahlungsprofil ohne die Peaks mit einem quadratischen Anstieg bis zu einem Maximum , gefolgt von einem Exponentiellen Abfall (\ref{fig:hintergrundkorrektur}). Die Peaks der Hintergrundstrahlung sind unproblematisch, da man sie immer bei den selben Energien findet.
	
	\begin{figure}
		\centering
		\includegraphics[width=1\linewidth]{Plots/Hintergrundkorrektur}
		\caption{}
		\label{fig:hintergrundkorrektur}
	\end{figure}
	
	\subsubsection{Energiekalibrierung}
	Bevor wir die Peaks der Hintergrundstrahlung analysieren müssen wir eine Energiekalibrierung für unseren Detektor durchführen, da der Zusammenhang zwischen Channelnummer und Energie nicht konstant, sondern stark Temperaturabhängig ist. Hierzu haben wir die Strahlungsquellen $^{60}Co$, $^{137}Cs$ und $^{241}Am$ verwendet Gammaspektren wir kennen. Wir wissen, dass die Energie mit der Channelnummer ansteigt und können die Peaks so einfach den Stoffen zuordnen.
	
	\begin{table}[H]
		\caption{Bekannte Energien}
		\centering
		\begin{tabular}{|c|c|}
			
			\hline
			Quelle & Energie \\
			\hline 
			$^{241}Am$ & 59.5409 keV \\ 
			\hline 
			$^{137}Cs$ & 661.7keV \\ 
			\hline 
			$^{60}Co$ &  1173.2,1332.5 keV\\ 
			\hline 
			
		\end{tabular} 
	\end{table}
	\begin{figure}
		\centering
		\includegraphics[width=1\linewidth]{Energiekalibrierung.png}
		\caption{Peak 5 gehört zum Pulser. Die restlichen Peaks sind die der Quellen  $^{60}Co$, $^{137}Cs$ und $^{241}Am$}
		\label{fig:hintergrund komplett}
	\end{figure}
	
	Damit können wir jetzt eine Energiekalibrierung durchführen. Wir nehmen einen quadratischen Zusammenhang zwischen der Channelnummer und der Energie an (\ref{fig:energiekalibrierung-fit}).
	\begin{equation}
	f(x)=ax^2+mx+b
	\end{equation}
	\begin{figure}
		\centering
		\includegraphics[width=1\linewidth]{Energiekalibrierung_Fit.png}
		\caption{Energiekalibrierung anhand der bekannten Energien.}
		\label{fig:energiekalibrierung-fit}
	\end{figure}
	
	\subsubsection{Zuordnung der Peaks}
	\paragraph{Hintergrundpeaks}\mbox{} \\
	Mithilfe der Energiekalibrierung können wir die Peaks der verschiedenen Stoffe analysieren. Dazu betrachten wir als erstes die Peaks der Hintergrundstrahlung um sie von den Peaks der Quellen unterscheiden zu können.
	\begin{figure}[H]
		\centering
		\includegraphics[width=1\linewidth]{hintergrund.png}
		\caption{}
		\label{fig:hintergrund}
	\end{figure}
	
	\begin{table}[H]
		\centering
		\begin{tabular}{|c|c|c|c|}
			\hline
			peak[keV] & peakerror[kev] & integrated peakevents &Wahrscheinliche Quelle\\ \hline\hline
			4.26e+02 & 4.42e-01 & 9.84e+02 &\\ \hline
			1.38e+03 & 4.81e-01 & 1.38e+03 & Co60\\ \hline
			3.53e+03 & 2.35e-01 & 1.15e+03 &\\ \hline
			3.84e+03 & 2.43e-01 & 1.44e+03 &\\ \hline
			6.81e+03 & 3.10e-01 & 1.55e+03 &\\ \hline
			7.74e+03 & 1.72e-01 & 1.57e+03 &\\ \hline
			8.48e+03 & 1.45e-01 & 2.75e+03 &\\ \hline
			1.38e+04 & 1.18e-02 & 1.80e+05 &\\ \hline
			
		\end{tabular}
		\caption{Hintergrundpeaks \label{tab:hintergrundpeaks}}
	\end{table}
	
	
	\subsubsection{Vergleich von Erwarteter und gemessener Zählrate und Effizienzkalibrierung}
	Wir kennen die Aktivität der Quellen und deren Halbwertszeit ($t_\frac{1}{2}$) zum Zeitpunkt des Erwerbs 1995. Die Aktivität zum Zeitpunkt des Experiments berechnet sich damit als
	\begin{equation}
	I(t)=I_0e^{\frac{ln(2)}{t_{1/2}}*t}
	\end{equation}
	Für unsere detektierten Intensitäten müssen wir die Energieabhängige Absorption durch das 3mm dicke Beryllium-Fenster vor dem Detektor berücksichtigen. Dazu verwenden wir das Beer-Lambert-Gesetzv und die Werte der Tabelle \ref{tab:mu}:
	\begin{equation}
	I(d,E)=I_0e^{-\frac{\mu(E)}{\rho}d\rho}
	\end{equation} 
	Man erkennt aus Abbildung (\ref{fig:abschwachung}), dass die Absorption nur für Energien unter 100keV eine Signifikante rolle spielt weshalb wir sie für die meisten Peaks vernachlässigen können.
	\begin{figure}
		\centering
		\includegraphics[width=0.7\linewidth]{Plots/Abschwachung}
		\caption{Abschwächung durch die 3mm dicke Berylliumplatte vor dem Detektor}
		\label{fig:abschwachung}
	\end{figure}
	
	Da unsere Strahlungsquellen nicht nur auf den Detektor, sondern annähernd isotrop Strahlen trifft nur ein kleiner Teil der Strahlung auf den Detektor. Da wir aber weder die größe des Detektors, noch den Abstand zwischen Detektor und Strahlungsquelle kennen, beschränken wir uns darauf, statt der Absoluten Effizienz des Detektors eine relative Effizienz $\epsilon_{rel}$ zu berechnen, die Proportional zur absoluten Effizienz $\epsilon_{abs}$ ist, wobei a die Proportionalitätskonstante ist.
	\begin{equation}
	\epsilon_{rel}=\epsilon_{abs}a=\frac{I(t)_{abs}a}{I(d,E)_{gemessen}}
	\end{equation}
	Die Relative energieabhängige Effizienz haben wir in Fig. \ref{fig:relintens} geplottet. Man erkennt klar, dass der Detektor für niedrige Energien sehr viel mehr Events erkennt, als für hohe Energien. Das liegt wahrscheinlich daran, dass für hohe Energien die Wahrscheinlichkeit sinkt, dass der gesamte Gammapeak durch den Photoeffekt absorbiert wird und Compton-Streuung eine dominante Rolle einnimmt, die sehr viel schlechter detektiert werden kann.
	\begin{figure}
		\centering
		\includegraphics[width=0.7\linewidth]{Plots/relintens.png}
		\caption{Man erkennt einen Kontinuierlichen Abfall der Effizienz des Detektors mit Ausnahme des einen Punktes bei 380keV, den wir als Messfehler deuten. }
		\label{fig:relintens}
	\end{figure}
	
	\begin{table}[H]
		\centering
		\begin{tabular}{|c|c|c|c|c|c|c|}
			\hline
			peak[keV]&$\Delta$peak[kev]&Literaturwer&Aktivität[Bq]&\%&Absolute Aktivität[kBq]&Quelle \\ \hline\hline
			4.77e+02 & 7.82e+00 & 5.11e+02  & 4.88e+00 &180&0.82& Na22\\ \hline
			1.27e+03 & 7.83e+00 & 1.27e+02& 1.10e+00 &100&0.82& Na22\\ \hline
			1.16e+03 & 7.81e+00 & 1.17e+03 & 2.75e+01 &100&20.25& Co60\\ \hline
			1.34e+03 & 7.81e+00 & 1.33+e03& 2.35e+01 &100&20.25& Co60\\ \hline
			6.26e+02 & 7.80e+00 & 6.61e+02 & 4.38e+02 &85&241.06& Ba137 Gamma\\ \hline
			7.85e+01 & 7.80e+00 & 8.00e+01 & 3.39e+02 &36&87.87& Ba133\\ \hline
			3.56e+02 & 7.80e+00 &3.56e+02 & 2.88e+01 &69&87.87& Ba133\\ \hline
			1.15e+02 & 7.80e+00 & 1.20e+02  & 4.16e+02 &37&142.59&Eu152\\ \hline
			3.19e+02 & 7.80e+00 & 3.44e+02 & 1.61e+02 &27&142.59&Eu152\\ \hline
			1.42e+03 & 7.81e+00 & 1.408e+03  & 3.22e+01 &22&142.59&Eu152\\ \hline
			5.96e+01 & 7.80e+00 &6.0e+01 & 1.49e+03 &36&379.60&Am241\\ \hline
			\hline
		\end{tabular}
		\caption{Tabelle mit allen Gammapeaks\label{}}
	\end{table}
	\subsubsection{Bestimmung der Unbekannten Probe}
	Zur Bestimmung der unbekannten Probe haben wir die Peaks aus dessen Spektrum bestimmt \ref{tab:unbekannt} und die Peaks anhand der Energie passenden Elementen zugeordnet. Zwei der Peaks stammen von der Co60 Probe die wahrscheinlich nicht ordentlich genug hinter den Bleiplatten verstaut wurde, oder trotzdem stark genug gestrahlt hat um die Messung zu beeinflussen. Der verbleibende Peak liegt bei 1480keV. Damit kommen K-40 mit einer Energie von 1460keV und Ni-65 mit einer Gammaenergie von 1481keV in Frage. Da Ni-65 eine Halbwertszeit von nur 2.56 Stunden hat und unsere Proben über 20 Jahre alt sind können wir diesen aber ausschließen. Es handelt sich also bei der unbekannten Probe um K-40 ($t_{halbe}=1.29*10^9$Jahre)
	Mithilfe der Effizientkalibrierung unseres Detektors könnten wir jetzt versuchen die Absolute Aktivität der K-40 Probe zu bestimmen. Dazu könnte man entweder den Wert der Effizienz bei 1420keV verwenden oder einen Linearen fit im hinteren Bereich der Kalibrierung vornehmen.
	Die Absolute Aktivität ist dann
	\begin{equation}
		\frac{A_{gemessen}}{\epsilon_{rel}}
	\end{equation}
	Da die Kaliumprobe aber in einer Flasche war und nicht wie die anderen Proben auf dem Detektor befestigt werden konnte ändert sich der Raumwinkel des Detektors und wir können keine Rückschlüsse auf die absolute Intensität machen.
	\subsubsection{Zuordnung der Hintergrundstrahlung zu einzelnen Elementen}
	Anhand der im Hintergrund gefundenen charakteristischen Peaks können wir versuchen die Elemente, die unsere Messung beeinflussen zu bestimmen \ref{tab:hintergrund}. Dabei fällt auf, dass wir die meisten unserer Proben auch im Hintergrund nachweisen können, obwohl diese durch Bleiplatten vom Detektor abschirmt waren.
	
	\begin{table}[H]
		\centering
		\begin{tabular}{|c|c|c|c|c|c|}
			\hline
			peak[keV]&$\Delta$peak[kev]&Literaturwert[keV]&peakevents&Aktivität[Bq]&Quelle \\ \hline\hline
			7.23e+01 & 7.86e+00 &Sm=70,Pt=77& 9.84e+02 & 2.74e-01 &  $\beta$-Zerfall,Sm-153,Pt-197\\ \hline
			2.21e+02 & 7.87e+00 &210& 1.38e+03 & 3.84e-01 & Ge77 \\ \hline
			5.74e+02 & 7.83e+00 &570& 1.15e+03 & 3.19e-01 & Bi-207 \\ \hline
			6.26e+02 & 7.83e+00 &661.64& 1.44e+03 & 4.01e-01 & Ba137(from Cs137)\\ \hline
			1.16e+03 & 7.84e+00 &1173.2& 1.55e+03 & 4.32e-01 &Co60\\ \hline
			1.34e+03 & 7.82e+00 &1332.5& 1.57e+03 & 4.37e-01 &Co60\\ \hline
			1.48e+03 & 7.82e+00 &1460& 2.75e+03 & 7.64e-01 & K40\\ \hline
			2.61e+03 & 7.80e+00 && 1.80e+05 & 5.00e+01 &Pulser \\ \hline
			\hline
		\end{tabular}
		\caption{\label{tab:hintergrund}}
	\end{table}
	Vor allem beim Peak bei 72keV und 574keV Peak sind wir uns unsicher woher diese Stammen. Es könnte sich beim ersten einerseits um $\beta$-Zerfälle handeln wie wir sie schon beim Cäsium gemessen haben, andererseites aber auch um Sm-153 oder Pt197 aufgrund der passenden Energien. Da beide Elemente eine kurze Halbwertszeit haben und relativ selten sind gehen wir davon aus, dass es sich um $\beta$-Peaks handelt.
	\subsubsection{Energieauflösung des Germaniumdetektors}
	Um die Energieauflösung unseres Detektors A zu bestimmen bilden wir das Verhätlnis aus der Energie und der Breite des Gaußpeaks. Dabei definieren wir die Breite des Peaks als $\sigma$. Man könnte hier auch die FWHM wählen die einfach nur $2\sqrt{2ln(2)}\sigma$ ist, je nachdem welches Kriterium man für die Unterscheidbarkeit zweier Peaks verwendet. 
	\begin{equation}
	A=\frac{\Delta E}{E}
	\end{equation}
	Hierbei vernachlässigen wir die Fehler für $\mu$ und $\sigma$, da diese von unserer Fitfunktion Teilweise schon größer sind, als die Werte selbst und eine Gaußsche Fehlerfortpflanzung daraufhin keinen Sinn mehr machen würde. Man erkennt in Abbildung \ref{fig:auflosungsvermogen}, dass das Auflösungsvermögen für steigende Energie immer besser wird. Das liegt daran, dass die Breite $\sigma$ unserer Peaks relativ konstant ist. Da unser Detektor und Diskreten Energien einen Kanal zuordnen kann ist es logisch, dass die Unschärfe durch das Digitalisieren des Signals relativ konstant ist. Außerdem befinden sich alle Proben bei Zimmertemperatur, weshalb wir das selbe Thermische Rauschen in allen Peaks messen.
	
	\begin{figure}
		\centering
		\includegraphics[width=0.7\linewidth]{Auflosungsvermogen}
		\caption{Auflösungsvermögen des Detektors für die Gamma-Peaks \ref{tab:Auflosung} \label{fig:auflosungsvermogen}
	\end{figure}
	
	
	\subsection{Peak To Total Verhältnis}
	Das Peak to total Verhältnis für die Cs137 Line können wir als Verhältnis der Integrierten Zählrate zur gesamten Zählrate bestimmen.
	\begin{equation}
	Ptt=\frac{N_{peak}}{N_{total}}
	\end{equation}
	\begin{equation}
	\Delta Ptt=\sqrt{ (\Delta N_{peak}/N_{total})^2+(N_{peak} \Delta N_{total}{N_{total}^2)^2}}
	\end{equation}
	Daraus ergibt sich für Ptt=0.1993 $\pm$ 0.00014
	Der Peak macht also ca 20\% der gesamten gemessenen Events des Spektrums aus.
	\subsection{Ergebnis}
	Wir konnten anhand der Energiekalibrierung die wir mit bekannten Proben durchgeführt haben die unbekannte Probe bestimmen, sowie Elemente aus der Hintergrundstrahlung, auch wenn es sich dabei meist um von unseren Quellen, die eigentlich abgeschirmt sein sollten handelt. Außerdem waren wir in der Lage eine energieabhängige Effizienzkalibrierung für unseren Detektor durchzuführen, mithilfe welcher man die absolute Aktivität für andere Proben bestimmen könnte, wenn man diese genauso auf dem Detektor positioniert wie Gruppe 23. Die von uns gemessenen Werte für die Gamma-Peaks weichen zwischen 10 und 50 keV vom Literaturwert ab. Während wir uns bei Energien ab 1000keV also im Bereich einer abweichung von 1-5 \% befinden, stellt das für niedrige Energien unter 100keV einen sehr großen Fehler da. Um diesen zu verringern könnte man die Energiekalibrierung mit mehr Proben durchführen. Da wir aber nur eine Probe (Am241) verwenden die eine sehr niedrige Gamma-Energie hat und diese Gleichzeitig zur Kallibration verwenden ist die Abweichung vom Literaturwert hierbei aber sehr gering. Die $\chi^2$ für die Fits unserer Plots befinden sich alle in einem Annehmbaren Bereich, wogegen die Fehler der Fitparameter teilweise größer sind als die Parameter selbst. Das deutet darauf hin, dass die Fehlerberechnung der Python Bibliothek lmfit nicht zuverlässig für unser Problem funktioniert, oder wir einen Fehler bei der Fehlerberechnung gemacht haben. Weiterhin bleibt anzumerken, dass wir für die Korrektur des Kontinuierlichen Hintergrundspektrums  zwei Funktionen verwendet haben, die dem Verlauf der Strahlung am besten zu beschreiben scheinen, ohne hierfür jedoch eine theoretische Begründung zu haben. Bei der Messung des Cs137 fällt auf, dass wir außer den Gammapeaks noch 2 Charakteristische $\beta$-Peaks messen die aus dem Zerfall von Cs137 zu Ba137 entstehen, wobei der Eigentlich zu messende $\gamma$-Peak vom Ba abgestrahlt wird. Eigentlich sollte unser Detektor durch die Berylliumscheibe von $\beta$-Strahlung abgeschirmt sein. Wenn man vermeiden wollte, dass unsere Messergebnisse also durch $\beta$-Strahlung beeinträchtigt werden, müsste man eine dickere Platte verwenden. Das hätte allerdings auch zur Folge, dass mehr $\gamma$-Strahlung herausgefiltert wird. Diese $\beta$-Strahlung findet sich auch in der Messung des Eu152 wieder wo wir charakteristische Peaks identifizieren können.	%genauigkeit unserer messungen
	%welche probleme beim messen
	%mögliche lösungen.
	
	
	\newpage
	\section{Anhang}
	
	\subsection{Na22}
	\begin{table}[H]
		\centering
		\begin{tabular}{|c|c|c|c|c|c|}
			\hline
			peak[keV]&$\Delta$peak[kev]&Literaturwert&peakevents&Aktivität[Bq]&Quelle \\ \hline\hline
			7.30e+01 & 7.89e+00 & & 4.71e+02 & 5.25e-01 & Hintergrund\\ \hline
			2.20e+02 & 7.91e+00 & &6.06e+02 & 6.74e-01 & Hintergrund\\ \hline
			4.77e+02 & 7.82e+00 & 5.11e+02 &4.39e+03 & 4.88e+00 & Na22\\ \hline
			1.27e+03 & 7.83e+00 & 1.27e+02&9.87e+02 & 1.10e+00 & Na22\\ \hline
			1.48e+03 & 7.84e+00 & &6.36e+02 & 7.07e-01 & K40\\ \hline
			2.61e+03 & 7.80e+00 & &4.49e+04 & 4.99e+01 & Pulser\\ \hline
			\hline
		\end{tabular}
		\caption{\label{}}
	\end{table}
	\subsection{Co60}
	\begin{table}[H]
		\centering
		\begin{tabular}{|c|c|c|c|c|c|}
			\hline
			peak[keV]&$\Delta$peak[kev]&Literaturwert&peakevents&Aktivität[Bq]&Quelle \\ \hline\hline
			1.16e+03 & 7.81e+00 & 1.17e+03& 2.46e+04 & 2.75e+01 & Co60\\ \hline
			1.34e+03 & 7.81e+00 & 1.33+e03& 2.10e+04 & 2.35e+01 & Co60\\ \hline
			2.61e+03 & 7.80e+00 && 4.46e+04 & 4.99e+01 & Pulser \\ \hline
			1.48e+03 & 7.83e+00 & 1.46e+03& 6.63e+02 & 7.41e-01 & K40\\ \hline
			\hline
		\end{tabular}
		\caption{\label{}}
	\end{table}
	\subsection{Unbekannt}
	\begin{table}[H]
		\centering
		\begin{tabular}{|c|c|c|c|c|c|}
			\hline
			peak[keV]&$\Delta$peak[kev]&Literaturwert&peakevents&Aktivität[Bq]&Quelle \\ \hline\hline
			1.16e+03 & 7.82e+00 & 1.17e+03 & 2.15e+00 &Hintergrund &Co60\\ \hline
			1.34e+03 & 7.80e+00 & 1.33e+03 & 4.99e+01 &Hintergrund &Co60\\ \hline
			1.48e+03 & 7.89e+00 & 1.46+e03 & 3.54e-01 &Hintergrund &K40\\ \hline
			\hline
		\end{tabular}
		\caption{\label{tab:unbekannt}}
	\end{table}
	\subsection{Cs137}
	\begin{table}[H]
		\centering
		\begin{tabular}{|c|c|c|c|c|c|}
			\hline
			peak[keV]&$\Delta$peak[kev]&Literaturwert&peakevents&Aktivität[Bq]&Quelle \\ \hline\hline
			3.54e+01 & 7.81e+00 && 1.36e+05 & 1.56e+02 & $\beta$-decay to Ba137\\ \hline
			3.93e+01 & 7.81e+00 && 3.55e+04 & 4.07e+01 & $\beta$-decay to Ba137\\ \hline
			6.26e+02 & 7.80e+00 & 6.61e+02& 3.82e+05 & 4.38e+02 & Ba137 Gamma\\ \hline
			2.61e+03 & 7.80e+00 && 4.18e+04 & 4.79e+01 & Pulser\\ \hline
			\hline
		\end{tabular}
		\caption{\label{}}
	\end{table}
	\subsection{Ba133}
	\begin{table}[H]
		\centering
		\begin{tabular}{|c|c|c|c|c|c|}
			\hline
			peak[keV]&$\Delta$peak[kev]&Literaturwert&peakevents&Aktivität[Bq]&Quelle \\ \hline\hline
			3.45e+01 & 7.83e+00 && 6.78e+05 & 7.77e+02 & $\beta$-decay Cs137 to Ba137 \\ \hline
			7.85e+01 & 7.80e+00 & 8.00e+01& 2.96e+05 & 3.39e+02 & Ba133\\ \hline
			2.55e+02 & 7.80e+00 && 2.93e+04 & 3.36e+01 & \\ \hline
			2.80e+02 & 7.80e+00 && 6.72e+04 & 7.70e+01 & \\ \hline
			3.30e+02 & 7.80e+00 && 1.91e+05 & 2.19e+02 & \\ \hline
			3.56e+02 & 7.80e+00 &3.56e+02& 2.51e+04 & 2.88e+01 & Ba133\\ \hline
			\hline
		\end{tabular}
		\caption{\label{}}
	\end{table}
	\subsection{Eu152}
	
	\begin{table}[H]
		\centering
		\begin{tabular}{|c|c|c|c|c|c|}
			\hline
			peak[keV]&$\Delta$peak[kev]&Literaturwert&peakevents&Aktivität[Bq]&Quelle \\ \hline\hline
			
			4.22e+01 & 7.84e+00 & &1.03e+06 & 1.21e+03 \\ \hline
			1.15e+02 & 7.80e+00 & 1.20e+02 &3.56e+05 & 4.16e+02 &Eu152\\ \hline
			2.26e+02 & 7.80e+00 & &5.34e+04 & 6.24e+01 &Hintergrund\\ \hline
			3.19e+02 & 7.80e+00 & 3.44e+02 &1.37e+05 & 1.61e+02 &Eu152\\ \hline
			4.72e+01 & 7.80e+00 &4.7e+01 &1.47e+05 & 1.72e+02 &Pb210\\ \hline
			3.82e+02 & 7.81e+00 & &1.80e+04 & 2.11e+01&\\ \hline
			4.13e+02 & 7.81e+00 & 4.11e+02 &2.07e+04 & 2.42e+01 &Eu152 $\beta$ \\ \hline
			7.44e+02 & 7.80e+00 & &3.65e+04 & 4.27e+01 &\\ \hline
			8.35e+02 & 7.81e+00 & &1.49e+04 & 1.74e+01 &\\ \hline
			9.36e+02 & 7.81e+00 & &3.07e+04 & 3.59e+01 &Hintergrund \\ \hline
			1.07e+03 & 7.84e+00 &1.09e+03 &2.19e+04 & 2.56e+01 &Eu152 $\beta$  \\ \hline
			1.09e+03 & 7.81e+00 & 1.11e+03 &2.56e+04 & 2.99e+01  &Eu152 $\beta$\\\hline
			1.42e+03 & 7.81e+00 & 1.408e+03 &2.76e+04 & 3.22e+01 &Eu152\\ \hline
			2.60e+03 & 7.80e+00 & &3.55e+04 & 4.15e+01 &Pulser \\ \hline
			\hline
		\end{tabular}
		\caption{\label{}}
	\end{table}
	
	\subsection{Am241}
	\begin{table}[H]
		\centering
		\begin{tabular}{|c|c|c|c|c|c|}
			\hline
			peak[keV]&$\Delta$peak[kev]&Literaturwert&peakevents&Aktivität[Bq]&Quelle \\ \hline\hline
			5.96e+01 & 7.80e+00 &6.0e+01& 1.31e+06 & 1.49e+03 &Am241\\ \hline
			2.21e+02 & 7.88e+00 && 4.15e+02 & 4.73e-01 & Hintergrund\\ \hline
			3.26e+02 & 7.87e+00 && 1.86e+02 & 2.12e-01 & \\ \hline
			6.26e+02 & 7.84e+00 &6.61e+02& 4.81e+02 & 5.49e-01 &Cs/Ba137 Gamma\\ \hline
			1.16e+03 & 7.86e+00& 1.17e+03& 4.32e+02 & 4.93e-01 &Co60\\ \hline
			1.48e+03 & 7.84e+00 &1.46+e03& 5.90e+02 & 6.73e-01 &K40\\ \hline
			2.61e+03 & 7.80e+00 && 4.37e+04 & 4.98e+01 &Pulser\\ \hline
			\hline
		\end{tabular}
		\caption{\label{}}
	\end{table}
\begin{table}[H]
	\centering
	\begin{tabular}{|c|c|}
		\hline
		Energie [keV] & Auflösungsvermögen \\ \hline\hline
		477.0 & 0.010209643605870022 \\ \hline
		1270.0 & 0.001921259842519685 \\ \hline
		1160.0 & 0.012672413793103448 \\ \hline
		1340.0 & 0.01171641791044776 \\ \hline
		626.0 & 0.2268370607028754 \\ \hline
		78.5 & 3.605095541401274 \\ \hline
		356.0 & 0.10056179775280898 \\ \hline
		115.0 & 0.6617391304347826 \\ \hline
		319.0 & 0.18369905956112853 \\ \hline
		1420.0 & 0.014295774647887324 \\ \hline
		59.6 & 16.493288590604028 \\ \hline
		\hline
	\end{tabular}
	\caption{Auflösungsvermögen \label{tab:Auflosung}}
\end{table}



	\newpage
	\begin{thebibliography}{}
		
		\bibitem{protonzerfall} \begin{verbatim}
		https://de.wikipedia.org/wiki/Protonenzerfall
		\end{verbatim}
		
		\bibitem{cobalt} \begin{verbatim}
		https://en.wikipedia.org/wiki/Cobalt-60
		\end{verbatim}
		
		\bibitem{energieaufloesung} 
		\begin{verbatim}
		Vorlesung2_11_2011.ppt , Torsten Kröll, TU Darmstadt
		\end{verbatim}  
		
		\bibitem{signalverarbeitung} 
		\begin{verbatim}
		Ge_Gamma_Spectroscopy.pdf , Canberra Nuclear, aus dem Wiki.
		\end{verbatim}  
		
	\end{thebibliography}
\end{document}

